
\documentclass[12pt,a4paper]{article}
\usepackage[utf8]{inputenc}
\usepackage[OT1]{fontenc}
\usepackage[russian]{babel}
\usepackage{amsmath}
\usepackage{amsfonts}
\usepackage{amssymb}
\usepackage{hyperref}
\hypersetup{linktoc=all}
\usepackage{indentfirst}
\usepackage[left=2cm,right=2cm,top=2cm,bottom=2cm]{geometry}
\usepackage{graphicx}
\usepackage{float}
\usepackage{pdflscape}
\usepackage{diagbox}
\usepackage{}
\author{Борисенков Никита Николаевич}
\title{Отчёт}
\date{}

\newcommand{\xo}{\mathring{x}}
\newcommand{\xx}{x\overline{x}}
\newcommand{\pd}[2]{\dfrac{\partial #1}{\partial #2}}
\DeclareMathOperator*{\mmax}{max}

\begin{document}
\maketitle
\tableofcontents
\newpage
\section{Постановка задачи}

Решается система уравнений, описывающая движение баротропного газа в двумерной области $\Omega$.

Эта система записывается следующим образом:

\begin{gather*}
    \pd{g}{t} + u_1 \pd{g}{x_1} + u_2 \pd{g}{x_2} + \pd{u_1}{x_1} \pd{u_2}{x_2}= 0\\
    \pd{u_k}{t} + u_1 \pd{u_k}{x_1} + u_2 \pd{u_k}{x_2} + p'_{\rho}(\rho) \pd{g}{x_k} =\\
    \dfrac{\mu}{\rho}\left( \dfrac43 \pd{^2 u_k}{x_k^2} + \sum_{m = 1, m \neq k}^2 \left(  \pd{^2 u_k}{x_m^2} + \dfrac13  \pd{^2 u_m}{x_k \partial x_m} \right) \right) + f_k, k = 1,2,\\
    p = p(\rho), g = \ln(\rho),
\end{gather*}

где $\rho$ это плотность газа, а $\textbf{u} = (u_1, u_2)$ вектор его скорости.
Область имеет вид $\Omega = \Omega_{00} \cup \Omega_{10} \cup \Omega_{11} \cup \Omega_{21}$, где $\Omega_{mn} = [m, m+1] \times [n, n+1]$.

\section{Алгоритм}

Для численного приближения решения системы используется последовательная схема с односторонними разностями для ($\ln(\rho), u$)

Уравнения, задающие схему выглядят следующим образом:

\begin{gather*}
    G_t + \delta_1\{\hat{G},V_1\} + \delta_2\{\hat{G},V_2\} + (V_1)_{\xo_1} + (V_2)_{\xo_2} = 0,  \textbf{x} \in \Omega_{\overline{h}}\\
    G_t + V_{kx_k} = 0, \textbf{x} \in \gamma^{-}_k, k = 1, 2; \\
    G_t + V_{k\overline{x}_k} = 0, \textbf{x} \in \gamma^{+}_k, k = 1, 2; \\
    \hat{H}(V_k)_t + \delta_1\{ \hat{V_k}, \hat{H}V_1 \} + \delta_2\{ \hat{V_k}, \hat{H}V_2 \} + p(\hat{H})_{\xo_k} = \\
    \mu\left( \dfrac43 (\hat{V}_k)_{x_k \overline{x}_k} + \sum_{m = 1, m \neq k}^2 (\hat{V}_k)_{x_m \overline{x}_m} + \dfrac13 \sum_{m = 1, m \neq k}^2 (V_m)_{\xo_k \xo_m} \right) + \hat{H}f_k, \textbf{x} \in \Omega_{\overline{h}}\\
    \hat{V_k} = 0, \textbf{x} \in \gamma_h, k = 1, 2
\end{gather*}

По этим уравнениям сначала строится СЛУ на значения $G$ на следующем слое, а с использованием полученных значений $\hat{H}$,
которые выражаются из $\hat{G}$ строится и решается СЛУ на $\hat{V_1}$ и $\hat{V_2}$.

После преобразований, СЛУ схемы имеют следующий вид:
\begin{gather*}
    G_{m_1, m_2}^{n+1}\left( 1 + \dfrac{\tau}{h}(\vert V_{m_1, m_2}^{n} \vert + \vert V_{m_1, m_2}^{n} \vert) \right) +\\
    + G_{m_1 - 1, m_2}^{n + 1} \dfrac{\tau}{2h} (-V_{1 m_1, m_2}^{n} - \vert V_{1 m_1, m_2}^{n}\vert) + G_{m_1 + 1, m_2}^{n + 1} \dfrac{\tau}{2h} (V_{1 m_1, m_2}^{n} - \vert V_{1 m_1, m_2}^{n}\vert) +\\
    + G_{m_1, m_2 - 1}^{n + 1} \dfrac{\tau}{2h} (-V_{2 m_1, m_2}^{n} - \vert V_{2 m_1, m_2}^{n}\vert) + G_{m_1, m_2 + 1}^{n + 1} \dfrac{\tau}{2h} (V_{2 m_1, m_2}^{n} - \vert V_{2 m_1, m_2}^{n}\vert) =\\
    G_{m_1, m_2}^{n} - \dfrac{\tau}{2h} ( V_{1 m_1 + 1, m_2}^{n} - V_{1 m_1 - 1, m_2}^{n} + V_{2 m_1, m_2 + 1}^{n} - V_{2 m_1, m_2 - 1}^{n} ), (m_1h, m_2h) \text{ -- внутренняя точка}  \\ 
    G_{m_1, m_2}^{n+1} = G_{m_1, m_2}^{n} - \dfrac{\tau}{h} ( V_{1 m_1 + 1, m_2}^{n} - V_{1 m_1, m_2}^{n}, (m_1h, m_2h) \text{ -- точка левой границы}  \\ 
    G_{m_1, m_2}^{n+1} = G_{m_1, m_2}^{n} - \dfrac{\tau}{h} ( V_{1 m_1, m_2}^{n} - V_{1 m_1 - 1, m_2}^{n}, (m_1h, m_2h) \text{ -- точка правой границы}  \\ 
    G_{m_1, m_2}^{n+1} = G_{m_1, m_2}^{n} - \dfrac{\tau}{h} ( V_{1 m_1, m_2 + 1}^{n} - V_{1 m_1, m_2}^{n}, (m_1h, m_2h) \text{ -- точка верхней границы}  \\ 
    G_{m_1, m_2}^{n+1} = G_{m_1, m_2}^{n} - \dfrac{\tau}{h} ( V_{1 m_1, m_2}^{n} - V_{1 m_1, m_2 - 1}^{n}, (m_1h, m_2h) \text{ -- точка нижней границы}  \\ 
    V_{1 m_1, m_2}^{n+1} \left(H_{m_1, m_2}^{n+1} + \dfrac{\tau}{h} \left( \vert H_{m_1, m_2}^{n+1} V_{1 m_1, m_2}^{n} \vert  + \vert H_{m_1, m_2}^{n+1} V_{2 m_1, m_2}^{n} \vert + \dfrac{14}{3} \dfrac{\mu}{h} \right) \right) +\\
    + V_{1 m_1 - 1, m_2}^{n + 1} \dfrac{\tau}{2h} \left(-H_{m_1, m_2}^{n+1}V_{1 m_1, m_2}^{n} - \vert H_{m_1, m_2}^{n+1}V_{1 m_1, m_2}^{n}\vert - \dfrac43 \dfrac{\mu}{h}\right) +\\
    + V_{1 m_1 + 1, m_2}^{n + 1} \dfrac{\tau}{2h} \left(H_{m_1, m_2}^{n+1}V_{1 m_1, m_2}^{n} - \vert H_{m_1, m_2}^{n+1}V_{1 m_1, m_2}^{n}\vert - \dfrac43 \dfrac{\mu}{h}\right) +\\
    + V_{1 m_1, m_2 - 1}^{n + 1} \dfrac{\tau}{2h} \left(-H_{m_1, m_2}^{n+1}V_{2 m_1, m_2}^{n} - \vert H_{m_1, m_2}^{n+1}V_{2 m_1, m_2}^{n}\vert - \dfrac{\mu}{h}\right) +\\
    + V_{1 m_1, m_2 + 1}^{n + 1} \dfrac{\tau}{2h} \left(H_{m_1, m_2}^{n+1}V_{2 m_1, m_2}^{n} - \vert H_{m_1, m_2}^{n+1}V_{2 m_1, m_2}^{n}\vert - \dfrac{\mu}{h}\right) =\\
    = H_{m_1, m_2}^{n+1}V_{1 m_1, m_2}^{n} - \tau \dfrac{1}{2h}(P(H_{m_1 + 1, m_2}^{n+1}) - P(H_{m_1 - 1, m_2}^{n+1})) +\\
    +\dfrac{1}{12} \dfrac{\tau\mu}{h^2} ( V_{2 m_1 + 1, m_2 + 1}^{n} - V_{2 m_1 + 1, m_2 + 1}^{n} - V_{2 m_1 - 1, m_2 + 1}^{n} + V_{2 m_1 - 1, m_2 - 1}^{n} ) + \tau H_{m_1, m_2}^{n+1}f_{1 m_1, m_2}^{n}\\
    V_{2 m_1, m_2}^{n+1} \left(H_{m_1, m_2}^{n+1} + \dfrac{\tau}{h} \left( \vert H_{m_1, m_2}^{n+1} V_{1 m_1, m_2}^{n} \vert  + \vert H_{m_1, m_2}^{n+1} V_{2 m_1, m_2}^{n} \vert + \dfrac{14}{3} \dfrac{\mu}{h} \right) \right) +\\
    + V_{1 m_1 - 1, m_2}^{n + 1} \dfrac{\tau}{2h} \left(-H_{m_1, m_2}^{n+1}V_{1 m_1, m_2}^{n} - \vert H_{m_1, m_2}^{n+1}V_{1 m_1, m_2}^{n}\vert - \dfrac{\mu}{h}\right) +\\
    + V_{1 m_1 + 1, m_2}^{n + 1} \dfrac{\tau}{2h} \left(H_{m_1, m_2}^{n+1}V_{1 m_1, m_2}^{n} - \vert H_{m_1, m_2}^{n+1}V_{1 m_1, m_2}^{n}\vert - \dfrac{\mu}{h}\right) +\\
    + V_{1 m_1, m_2 - 1}^{n + 1} \dfrac{\tau}{2h} \left(-H_{m_1, m_2}^{n+1}V_{2 m_1, m_2}^{n} - \vert H_{m_1, m_2}^{n+1}V_{2 m_1, m_2}^{n}\vert - \dfrac43 \dfrac{\mu}{h}\right) +\\
    + V_{1 m_1, m_2 + 1}^{n + 1} \dfrac{\tau}{2h} \left(H_{m_1, m_2}^{n+1}V_{2 m_1, m_2}^{n} - \vert H_{m_1, m_2}^{n+1}V_{2 m_1, m_2}^{n}\vert - \dfrac43 \dfrac{\mu}{h}\right) =\\
    = H_{m_1, m_2}^{n+1}V_{2 m_1, m_2}^{n} - \tau \dfrac{1}{2h}(P(H_{m_1 + 1, m_2}^{n+1}) - P(H_{m_1 - 1, m_2}^{n+1})) +\\
    +\dfrac{1}{12} \dfrac{\tau\mu}{h^2} ( V_{1 m_1 + 1, m_2 + 1}^{n} - V_{1 m_1 + 1, m_2 + 1}^{n} - V_{1 m_1 - 1, m_2 + 1}^{n} + V_{1 m_1 - 1, m_2 - 1}^{n} ) + \tau H_{m_1, m_2}^{n+1}f_{1 m_1, m_2}^{n}\\
    V_{km_1, m_2}^{n+1} = 0, k = 1,2 \text{ для граничных точек}
\end{gather*}
%\section{Отладочный тест}

Для проверки правильности написанной схемы, она применялась с начальными условиями и правыми частями, соответствующими следующим гладким решениям:
\begin{equation*}
    \rho(t, x) = e^{t + x_1 - x_2}, u_1(t, x) = \sin(2\pi x_1)\sin(2\pi x_2)e^{-t}, u_2 = \sin(2\pi x_1)\sin(2\pi x_2)e^{t}
\end{equation*}

\subsection{Результаты на разных параметрах}

В следующих таблицах приведены результаты работы программы при разных значениях параметров схемы и задачи.

\subsubsection{$\mu = 0.1, p(\rho) = \rho$}
$$\Vert g - \ln(\rho)\Vert$$
\begin{tabular}{*{5}{|c}|}
\hline
\diagbox{$\tau$}{$h$}&0.050000&0.025000&0.012500&0.0062500\\
\hline
0.050000&$1.088e+00$&$4.445e-01$&$1.976e-01$&$5.384e+00$\\
&$4.949e-01$&$2.554e-01$&$1.511e-01$&$1.396e-01$\\
&$5.563e-01$&$2.628e-01$&$1.519e-01$&$1.554e-01$\\
&$0.084$&$0.475$&$9.120$&$240.998$\\
\hline
0.025000&$1.258e+00$&$5.310e-01$&$2.511e-01$&$1.162e-01$\\
&$4.917e-01$&$2.435e-01$&$1.322e-01$&$7.851e-02$\\
&$5.660e-01$&$2.535e-01$&$1.334e-01$&$7.863e-02$\\
&$0.091$&$0.651$&$11.220$&$270.404$\\
\hline
0.012500&$1.350e+00$&$5.788e-01$&$2.820e-01$&$1.370e-01$\\
&$4.920e-01$&$2.397e-01$&$1.252e-01$&$6.808e-02$\\
&$5.738e-01$&$2.515e-01$&$1.269e-01$&$6.828e-02$\\
&$0.153$&$0.994$&$16.700$&$347.908$\\
\hline
0.0062500&$1.398e+00$&$6.041e-01$&$2.985e-01$&$1.488e-01$\\
&$4.928e-01$&$2.385e-01$&$1.228e-01$&$6.420e-02$\\
&$5.786e-01$&$2.513e-01$&$1.247e-01$&$6.446e-02$\\
&$0.317$&$1.488$&$20.392$&$360.586$\\
\hline
0.0031250&$1.422e+00$&$6.171e-01$&$3.070e-01$&$1.550e-01$\\
&$4.933e-01$&$2.381e-01$&$1.218e-01$&$6.273e-02$\\
&$5.813e-01$&$2.515e-01$&$1.239e-01$&$6.303e-02$\\
&$0.469$&$2.649$&$30.966$&$442.823$\\
\hline
0.0015625&$1.435e+00$&$6.237e-01$&$3.114e-01$&$1.582e-01$\\
&$4.936e-01$&$2.380e-01$&$1.214e-01$&$6.213e-02$\\
&$5.827e-01$&$2.516e-01$&$1.236e-01$&$6.245e-02$\\
&$0.839$&$4.166$&$63.827$&$525.864$\\
\hline
\end{tabular}
$$\Vert v_1 - u_1 \Vert$$
\begin{tabular}{*{5}{|c}|}
\hline
\diagbox{$\tau$}{$h$}&0.050000&0.025000&0.012500&0.0062500\\
\hline
0.050000&$3.940e-01$&$2.222e-01$&$1.425e-01$&$5.857e-01$\\
&$2.220e-01$&$1.274e-01$&$7.943e-02$&$6.542e-02$\\
&$2.538e-01$&$1.329e-01$&$8.031e-02$&$6.719e-02$\\
&$0.084$&$0.475$&$9.120$&$240.998$\\
\hline
0.025000&$4.209e-01$&$2.309e-01$&$1.210e-01$&$7.169e-02$\\
&$2.166e-01$&$1.188e-01$&$6.644e-02$&$4.063e-02$\\
&$2.494e-01$&$1.248e-01$&$6.737e-02$&$4.076e-02$\\
&$0.091$&$0.651$&$11.220$&$270.404$\\
\hline
0.012500&$4.311e-01$&$2.363e-01$&$1.241e-01$&$6.378e-02$\\
&$2.147e-01$&$1.159e-01$&$6.198e-02$&$3.399e-02$\\
&$2.483e-01$&$1.223e-01$&$6.301e-02$&$3.413e-02$\\
&$0.153$&$0.994$&$16.700$&$347.908$\\
\hline
0.0062500&$4.356e-01$&$2.396e-01$&$1.263e-01$&$6.519e-02$\\
&$2.140e-01$&$1.148e-01$&$6.044e-02$&$3.172e-02$\\
&$2.481e-01$&$1.215e-01$&$6.153e-02$&$3.187e-02$\\
&$0.317$&$1.488$&$20.392$&$360.586$\\
\hline
0.0031250&$4.378e-01$&$2.414e-01$&$1.277e-01$&$6.605e-02$\\
&$2.137e-01$&$1.144e-01$&$5.986e-02$&$3.092e-02$\\
&$2.480e-01$&$1.211e-01$&$6.099e-02$&$3.108e-02$\\
&$0.469$&$2.649$&$30.966$&$442.823$\\
\hline
0.0015625&$4.389e-01$&$2.423e-01$&$1.285e-01$&$6.652e-02$\\
&$2.135e-01$&$1.142e-01$&$5.962e-02$&$3.062e-02$\\
&$2.480e-01$&$1.210e-01$&$6.077e-02$&$3.079e-02$\\
&$0.839$&$4.166$&$63.827$&$525.864$\\
\hline
\end{tabular}
$$\Vert v_2 - u_2\Vert$$
\begin{tabular}{*{5}{|c}|}
\hline
\diagbox{$\tau$}{$h$}&0.050000&0.025000&0.012500&0.0062500\\
\hline
0.050000&$8.106e-01$&$4.844e-01$&$2.592e-01$&$6.827e-01$\\
&$4.315e-01$&$2.550e-01$&$1.418e-01$&$9.679e-02$\\
&$4.932e-01$&$2.650e-01$&$1.431e-01$&$9.843e-02$\\
&$0.084$&$0.475$&$9.120$&$240.998$\\
\hline
0.025000&$9.106e-01$&$5.588e-01$&$3.008e-01$&$1.503e-01$\\
&$4.498e-01$&$2.672e-01$&$1.445e-01$&$7.566e-02$\\
&$5.168e-01$&$2.784e-01$&$1.460e-01$&$7.585e-02$\\
&$0.091$&$0.651$&$11.220$&$270.404$\\
\hline
0.012500&$9.650e-01$&$6.005e-01$&$3.363e-01$&$1.741e-01$\\
&$4.606e-01$&$2.759e-01$&$1.503e-01$&$7.713e-02$\\
&$5.307e-01$&$2.878e-01$&$1.521e-01$&$7.736e-02$\\
&$0.153$&$0.994$&$16.700$&$347.908$\\
\hline
0.0062500&$9.934e-01$&$6.226e-01$&$3.557e-01$&$1.884e-01$\\
&$4.664e-01$&$2.808e-01$&$1.544e-01$&$7.994e-02$\\
&$5.381e-01$&$2.933e-01$&$1.562e-01$&$8.019e-02$\\
&$0.317$&$1.488$&$20.392$&$360.586$\\
\hline
0.0031250&$1.008e+00$&$6.339e-01$&$3.657e-01$&$1.960e-01$\\
&$4.694e-01$&$2.835e-01$&$1.566e-01$&$8.184e-02$\\
&$5.419e-01$&$2.961e-01$&$1.586e-01$&$8.211e-02$\\
&$0.469$&$2.649$&$30.966$&$442.823$\\
\hline
0.0015625&$1.015e+00$&$6.397e-01$&$3.708e-01$&$2.000e-01$\\
&$4.710e-01$&$2.849e-01$&$1.579e-01$&$8.290e-02$\\
&$5.439e-01$&$2.976e-01$&$1.598e-01$&$8.318e-02$\\
&$0.839$&$4.166$&$63.827$&$525.864$\\
\hline
\end{tabular}


\subsubsection{$\mu = 0.01, p(\rho) = \rho$}
$$\Vert g - \ln(\rho)\Vert$$
\begin{tabular}{*{5}{|c}|}
\hline
\diagbox{$\tau$}{$h$}&0.050000&0.025000&0.012500&0.0062500\\
\hline
0.050000&$1.792e+00$&$7.973e-01$&$5.209e+00$&$2.638e+11$\\
&$6.916e-01$&$3.546e-01$&$2.152e+00$&$6.160e+10$\\
&$8.232e-01$&$3.785e-01$&$2.224e+00$&$6.666e+10$\\
&$0.055$&$0.393$&$7.223$&$218.758$\\
\hline
0.025000&$1.948e+00$&$9.868e-01$&$4.912e-01$&$5.584e+64$\\
&$7.106e-01$&$3.686e-01$&$1.982e-01$&$2.870e+63$\\
&$8.670e-01$&$4.009e-01$&$2.061e-01$&$5.439e+63$\\
&$0.085$&$0.784$&$9.564$&$240.243$\\
\hline
0.012500&$2.031e+00$&$1.081e+00$&$5.825e-01$&$4.186e-01$\\
&$7.212e-01$&$3.822e-01$&$2.043e-01$&$1.075e-01$\\
&$8.885e-01$&$4.193e-01$&$2.111e-01$&$1.109e-01$\\
&$0.149$&$0.990$&$9.968$&$258.075$\\
\hline
0.0062500&$2.074e+00$&$1.129e+00$&$6.345e-01$&$3.063e-01$\\
&$7.270e-01$&$3.906e-01$&$2.103e-01$&$1.079e-01$\\
&$9.002e-01$&$4.301e-01$&$2.178e-01$&$1.091e-01$\\
&$0.239$&$1.264$&$12.073$&$281.971$\\
\hline
0.0031250&$2.096e+00$&$1.153e+00$&$6.616e-01$&$3.208e-01$\\
&$7.305e-01$&$3.952e-01$&$2.140e-01$&$1.100e-01$\\
&$9.069e-01$&$4.359e-01$&$2.219e-01$&$1.113e-01$\\
&$0.471$&$2.034$&$15.821$&$288.972$\\
\hline
0.0015625&$2.107e+00$&$1.165e+00$&$6.755e-01$&$3.283e-01$\\
&$7.324e-01$&$3.976e-01$&$2.160e-01$&$1.114e-01$\\
&$9.103e-01$&$4.388e-01$&$2.241e-01$&$1.127e-01$\\
&$0.855$&$3.585$&$26.889$&$331.247$\\
\hline
\end{tabular}
$$\Vert v_1 - u_1 \Vert$$
\begin{tabular}{*{5}{|c}|}
\hline
\diagbox{$\tau$}{$h$}&0.050000&0.025000&0.012500&0.0062500\\
\hline
0.050000&$8.893e-01$&$6.627e-01$&$1.547e+01$&$3.679e-01$\\
&$4.108e-01$&$2.546e-01$&$1.209e+00$&$3.679e-01$\\
&$5.161e-01$&$2.860e-01$&$1.519e+00$&$3.684e-01$\\
&$0.055$&$0.393$&$7.223$&$218.758$\\
\hline
0.025000&$9.139e-01$&$7.130e-01$&$4.622e-01$&$3.679e-01$\\
&$4.207e-01$&$2.631e-01$&$1.552e-01$&$3.679e-01$\\
&$5.354e-01$&$2.983e-01$&$1.660e-01$&$3.684e-01$\\
&$0.085$&$0.784$&$9.564$&$240.243$\\
\hline
0.012500&$9.392e-01$&$7.434e-01$&$4.980e-01$&$3.063e-01$\\
&$4.265e-01$&$2.699e-01$&$1.599e-01$&$8.772e-02$\\
&$5.443e-01$&$3.074e-01$&$1.696e-01$&$9.077e-02$\\
&$0.149$&$0.990$&$9.968$&$258.075$\\
\hline
0.0062500&$9.739e-01$&$7.584e-01$&$5.167e-01$&$3.293e-01$\\
&$4.300e-01$&$2.739e-01$&$1.636e-01$&$8.996e-02$\\
&$5.490e-01$&$3.126e-01$&$1.737e-01$&$9.191e-02$\\
&$0.239$&$1.264$&$12.073$&$281.971$\\
\hline
0.0031250&$1.001e+00$&$7.656e-01$&$5.262e-01$&$3.406e-01$\\
&$4.317e-01$&$2.761e-01$&$1.657e-01$&$9.195e-02$\\
&$5.516e-01$&$3.154e-01$&$1.760e-01$&$9.397e-02$\\
&$0.471$&$2.034$&$15.821$&$288.972$\\
\hline
0.0015625&$1.012e+00$&$7.692e-01$&$5.309e-01$&$3.462e-01$\\
&$4.327e-01$&$2.772e-01$&$1.668e-01$&$9.304e-02$\\
&$5.530e-01$&$3.168e-01$&$1.772e-01$&$9.510e-02$\\
&$0.855$&$3.585$&$26.889$&$331.247$\\
\hline
\end{tabular}
$$\Vert v_2 - u_2\Vert$$
\begin{tabular}{*{5}{|c}|}
\hline
\diagbox{$\tau$}{$h$}&0.050000&0.025000&0.012500&0.0062500\\
\hline
0.050000&$1.810e+00$&$1.245e+00$&$2.703e+00$&$2.718e+00$\\
&$8.762e-01$&$5.343e-01$&$1.183e+00$&$2.718e+00$\\
&$1.048e+00$&$5.675e-01$&$1.215e+00$&$2.722e+00$\\
&$0.055$&$0.393$&$7.223$&$218.758$\\
\hline
0.025000&$2.233e+00$&$1.443e+00$&$8.806e-01$&$2.718e+00$\\
&$9.181e-01$&$5.758e-01$&$3.205e-01$&$2.718e+00$\\
&$1.107e+00$&$6.143e-01$&$3.276e-01$&$2.722e+00$\\
&$0.085$&$0.784$&$9.564$&$240.243$\\
\hline
0.012500&$2.513e+00$&$1.539e+00$&$1.001e+00$&$5.105e-01$\\
&$9.453e-01$&$5.995e-01$&$3.419e-01$&$1.771e-01$\\
&$1.149e+00$&$6.408e-01$&$3.493e-01$&$1.788e-01$\\
&$0.149$&$0.990$&$9.968$&$258.075$\\
\hline
0.0062500&$2.574e+00$&$1.581e+00$&$1.065e+00$&$5.646e-01$\\
&$9.583e-01$&$6.119e-01$&$3.545e-01$&$1.864e-01$\\
&$1.167e+00$&$6.546e-01$&$3.624e-01$&$1.876e-01$\\
&$0.239$&$1.264$&$12.073$&$281.971$\\
\hline
0.0031250&$2.583e+00$&$1.600e+00$&$1.098e+00$&$5.944e-01$\\
&$9.657e-01$&$6.182e-01$&$3.613e-01$&$1.924e-01$\\
&$1.176e+00$&$6.616e-01$&$3.694e-01$&$1.936e-01$\\
&$0.471$&$2.034$&$15.821$&$288.972$\\
\hline
0.0015625&$2.584e+00$&$1.609e+00$&$1.115e+00$&$6.098e-01$\\
&$9.699e-01$&$6.213e-01$&$3.647e-01$&$1.955e-01$\\
&$1.182e+00$&$6.652e-01$&$3.730e-01$&$1.968e-01$\\
&$0.855$&$3.585$&$26.889$&$331.247$\\
\hline
\end{tabular}


\subsubsection{$\mu = 0.001, p(\rho) = \rho$}
$$\Vert g - \ln(\rho)\Vert$$
\begin{tabular}{*{5}{|c}|}
\hline
\diagbox{$\tau$}{$h$}&0.050000&0.025000&0.012500&0.0062500\\
\hline
0.050000&$2.117e+00$&$4.554e+00$&$7.321e+00$&$3.000e+00$\\
&$7.585e-01$&$3.458e+00$&$3.622e+00$&$3.266e+00$\\
&$9.206e-01$&$3.480e+00$&$3.851e+00$&$3.266e+00$\\
&$0.048$&$0.601$&$3.983$&$237.034$\\
\hline
0.025000&$2.307e+00$&$1.263e+00$&$8.815e+19$&$4.937e+47$\\
&$7.796e-01$&$4.310e-01$&$9.454e+18$&$3.628e+45$\\
&$9.655e-01$&$4.772e-01$&$1.646e+19$&$6.284e+45$\\
&$0.075$&$0.762$&$10.506$&$124.443$\\
\hline
0.012500&$2.410e+00$&$1.387e+00$&$8.963e-01$&$1.496e+70$\\
&$7.932e-01$&$4.536e-01$&$2.500e-01$&$2.427e+68$\\
&$9.895e-01$&$5.050e-01$&$2.607e-01$&$4.039e+68$\\
&$0.135$&$1.092$&$7.391$&$208.692$\\
\hline
0.0062500&$2.465e+00$&$1.454e+00$&$9.573e-01$&$4.284e-01$\\
&$8.018e-01$&$4.661e-01$&$2.616e-01$&$1.321e-01$\\
&$1.004e+00$&$5.200e-01$&$2.734e-01$&$1.340e-01$\\
&$0.238$&$1.162$&$10.214$&$240.822$\\
\hline
0.0031250&$2.492e+00$&$1.489e+00$&$9.895e-01$&$4.529e-01$\\
&$8.065e-01$&$4.726e-01$&$2.681e-01$&$1.364e-01$\\
&$1.011e+00$&$5.277e-01$&$2.804e-01$&$1.384e-01$\\
&$0.441$&$1.993$&$10.712$&$241.361$\\
\hline
0.0015625&$2.506e+00$&$1.506e+00$&$1.007e+00$&$4.653e-01$\\
&$8.090e-01$&$4.758e-01$&$2.716e-01$&$1.389e-01$\\
&$1.016e+00$&$5.316e-01$&$2.841e-01$&$1.410e-01$\\
&$0.849$&$3.708$&$20.682$&$188.010$\\
\hline
\end{tabular}
$$\Vert v_1 - u_1 \Vert$$
\begin{tabular}{*{5}{|c}|}
\hline
\diagbox{$\tau$}{$h$}&0.050000&0.025000&0.012500&0.0062500\\
\hline
0.050000&$9.817e-01$&$9.039e-01$&$4.461e+04$&$5.213e-01$\\
&$4.709e-01$&$7.868e-01$&$2.379e+03$&$4.215e-01$\\
&$6.081e-01$&$8.304e-01$&$3.308e+03$&$4.223e-01$\\
&$0.048$&$0.601$&$3.983$&$237.034$\\
\hline
0.025000&$1.036e+00$&$8.759e-01$&$3.679e-01$&$3.679e-01$\\
&$4.845e-01$&$3.223e-01$&$3.679e-01$&$3.679e-01$\\
&$6.276e-01$&$3.766e-01$&$3.701e-01$&$3.684e-01$\\
&$0.075$&$0.762$&$10.506$&$124.443$\\
\hline
0.012500&$1.114e+00$&$9.176e-01$&$6.561e-01$&$3.679e-01$\\
&$4.923e-01$&$3.319e-01$&$2.120e-01$&$3.679e-01$\\
&$6.387e-01$&$3.894e-01$&$2.314e-01$&$3.684e-01$\\
&$0.135$&$1.092$&$7.391$&$208.692$\\
\hline
0.0062500&$1.138e+00$&$9.453e-01$&$6.677e-01$&$5.116e-01$\\
&$4.963e-01$&$3.373e-01$&$2.171e-01$&$1.293e-01$\\
&$6.441e-01$&$3.965e-01$&$2.372e-01$&$1.348e-01$\\
&$0.238$&$1.162$&$10.214$&$240.822$\\
\hline
0.0031250&$1.149e+00$&$9.604e-01$&$6.785e-01$&$5.194e-01$\\
&$4.987e-01$&$3.402e-01$&$2.199e-01$&$1.324e-01$\\
&$6.472e-01$&$4.002e-01$&$2.403e-01$&$1.380e-01$\\
&$0.441$&$1.993$&$10.712$&$241.361$\\
\hline
0.0015625&$1.154e+00$&$9.681e-01$&$6.837e-01$&$5.225e-01$\\
&$5.001e-01$&$3.417e-01$&$2.214e-01$&$1.341e-01$\\
&$6.491e-01$&$4.021e-01$&$2.419e-01$&$1.398e-01$\\
&$0.849$&$3.708$&$20.682$&$188.010$\\
\hline
\end{tabular}
$$\Vert v_2 - u_2\Vert$$
\begin{tabular}{*{5}{|c}|}
\hline
\diagbox{$\tau$}{$h$}&0.050000&0.025000&0.012500&0.0062500\\
\hline
0.050000&$2.338e+00$&$3.202e+00$&$2.389e+02$&$3.200e+00$\\
&$9.951e-01$&$2.200e+00$&$4.410e+00$&$2.743e+00$\\
&$1.216e+00$&$2.281e+00$&$6.914e+00$&$2.748e+00$\\
&$0.048$&$0.601$&$3.983$&$237.034$\\
\hline
0.025000&$2.633e+00$&$1.774e+00$&$2.718e+00$&$2.718e+00$\\
&$1.042e+00$&$6.793e-01$&$2.718e+00$&$2.718e+00$\\
&$1.285e+00$&$7.351e-01$&$2.735e+00$&$2.722e+00$\\
&$0.075$&$0.762$&$10.506$&$124.443$\\
\hline
0.012500&$2.683e+00$&$1.862e+00$&$1.338e+00$&$2.718e+00$\\
&$1.072e+00$&$7.055e-01$&$4.231e-01$&$2.718e+00$\\
&$1.326e+00$&$7.656e-01$&$4.361e-01$&$2.722e+00$\\
&$0.135$&$1.092$&$7.391$&$208.692$\\
\hline
0.0062500&$2.726e+00$&$1.898e+00$&$1.412e+00$&$8.025e-01$\\
&$1.095e+00$&$7.189e-01$&$4.391e-01$&$2.377e-01$\\
&$1.360e+00$&$7.816e-01$&$4.530e-01$&$2.399e-01$\\
&$0.238$&$1.162$&$10.214$&$240.822$\\
\hline
0.0031250&$2.733e+00$&$1.958e+00$&$1.450e+00$&$8.396e-01$\\
&$1.106e+00$&$7.258e-01$&$4.475e-01$&$2.456e-01$\\
&$1.377e+00$&$7.900e-01$&$4.618e-01$&$2.480e-01$\\
&$0.441$&$1.993$&$10.712$&$241.361$\\
\hline
0.0015625&$2.734e+00$&$2.020e+00$&$1.471e+00$&$8.588e-01$\\
&$1.112e+00$&$7.292e-01$&$4.518e-01$&$2.498e-01$\\
&$1.385e+00$&$7.941e-01$&$4.664e-01$&$2.523e-01$\\
&$0.849$&$3.708$&$20.682$&$188.010$\\
\hline
\end{tabular}


\subsubsection{$\mu = 0.1, p(\rho) = 10\rho$}
$$\Vert g - \ln(\rho)\Vert$$
\begin{tabular}{*{5}{|c}|}
\hline
\diagbox{$\tau$}{$h$}&0.050000&0.025000&0.012500&0.0062500\\
\hline
0.050000&$8.428e+01$&$3.519e+05$&$2.935e+08$&$2.130e+04$\\
&$6.561e+00$&$3.893e+04$&$1.805e+08$&$6.425e+02$\\
&$8.161e+00$&$4.952e+04$&$1.832e+08$&$7.244e+02$\\
&$0.081$&$0.865$&$12.578$&$308.577$\\
\hline
0.025000&$1.272e-01$&$1.917e+03$&$4.468e+46$&$1.397e+03$\\
&$1.510e-01$&$3.880e+01$&$1.236e+46$&$9.859e+00$\\
&$1.554e-01$&$8.132e+01$&$1.947e+46$&$1.609e+01$\\
&$0.105$&$1.129$&$9.905$&$362.942$\\
\hline
0.012500&$1.268e-01$&$7.121e-02$&$2.987e+00$&$1.998e+09$\\
&$1.509e-01$&$7.734e-02$&$2.449e+00$&$1.628e+08$\\
&$1.555e-01$&$7.802e-02$&$2.450e+00$&$2.601e+08$\\
&$0.164$&$1.302$&$23.119$&$385.953$\\
\hline
0.0062500&$1.266e-01$&$7.072e-02$&$3.796e-02$&$5.924e+01$\\
&$1.508e-01$&$7.731e-02$&$4.031e-02$&$4.027e-01$\\
&$1.556e-01$&$7.802e-02$&$4.040e-02$&$5.865e-01$\\
&$0.270$&$1.518$&$23.660$&$410.648$\\
\hline
0.0031250&$1.264e-01$&$7.046e-02$&$3.763e-02$&$1.961e-02$\\
&$1.508e-01$&$7.730e-02$&$4.027e-02$&$2.073e-02$\\
&$1.556e-01$&$7.802e-02$&$4.037e-02$&$2.075e-02$\\
&$0.487$&$2.468$&$31.381$&$473.422$\\
\hline
0.0015625&$1.263e-01$&$7.033e-02$&$3.747e-02$&$1.943e-02$\\
&$1.508e-01$&$7.730e-02$&$4.025e-02$&$2.070e-02$\\
&$1.557e-01$&$7.803e-02$&$4.035e-02$&$2.071e-02$\\
&$0.898$&$4.232$&$45.597$&$524.872$\\
\hline
\end{tabular}
$$\Vert v_1 - u_1 \Vert$$
\begin{tabular}{*{5}{|c}|}
\hline
\diagbox{$\tau$}{$h$}&0.050000&0.025000&0.012500&0.0062500\\
\hline
0.050000&$1.384e+01$&$1.112e+01$&$3.679e-01$&$3.802e+00$\\
&$4.858e+00$&$3.859e+00$&$3.679e-01$&$3.620e+00$\\
&$5.362e+00$&$3.991e+00$&$3.701e-01$&$3.622e+00$\\
&$0.081$&$0.865$&$12.578$&$308.577$\\
\hline
0.025000&$6.021e-01$&$5.666e+00$&$3.679e-01$&$8.355e+01$\\
&$3.376e-01$&$3.716e+00$&$3.679e-01$&$3.265e+00$\\
&$3.906e-01$&$3.794e+00$&$3.701e-01$&$3.388e+00$\\
&$0.105$&$1.129$&$9.905$&$362.942$\\
\hline
0.012500&$6.052e-01$&$3.639e-01$&$3.154e+00$&$3.679e-01$\\
&$3.356e-01$&$1.883e-01$&$2.198e+00$&$3.679e-01$\\
&$3.896e-01$&$1.982e-01$&$2.205e+00$&$3.684e-01$\\
&$0.164$&$1.302$&$23.119$&$385.953$\\
\hline
0.0062500&$6.068e-01$&$3.646e-01$&$1.991e-01$&$2.525e+00$\\
&$3.348e-01$&$1.871e-01$&$1.002e-01$&$6.366e-01$\\
&$3.892e-01$&$1.971e-01$&$1.018e-01$&$6.498e-01$\\
&$0.270$&$1.518$&$23.660$&$410.648$\\
\hline
0.0031250&$6.076e-01$&$3.650e-01$&$1.994e-01$&$1.054e-01$\\
&$3.344e-01$&$1.865e-01$&$9.953e-02$&$5.186e-02$\\
&$3.891e-01$&$1.967e-01$&$1.011e-01$&$5.208e-02$\\
&$0.487$&$2.468$&$31.381$&$473.422$\\
\hline
0.0015625&$6.080e-01$&$3.653e-01$&$1.995e-01$&$1.053e-01$\\
&$3.342e-01$&$1.862e-01$&$9.922e-02$&$5.152e-02$\\
&$3.890e-01$&$1.964e-01$&$1.008e-01$&$5.175e-02$\\
&$0.898$&$4.232$&$45.597$&$524.872$\\
\hline
\end{tabular}
$$\Vert v_2 - u_2\Vert$$
\begin{tabular}{*{5}{|c}|}
\hline
\diagbox{$\tau$}{$h$}&0.050000&0.025000&0.012500&0.0062500\\
\hline
0.050000&$4.784e+00$&$3.944e+03$&$2.718e+00$&$4.742e+00$\\
&$1.994e+00$&$4.803e+02$&$2.718e+00$&$3.215e+00$\\
&$2.369e+00$&$5.787e+02$&$2.735e+00$&$3.218e+00$\\
&$0.081$&$0.865$&$12.578$&$308.577$\\
\hline
0.025000&$6.661e-01$&$9.620e+05$&$2.718e+00$&$4.130e+00$\\
&$3.545e-01$&$3.026e+04$&$2.718e+00$&$1.491e+00$\\
&$4.071e-01$&$5.149e+04$&$2.735e+00$&$1.498e+00$\\
&$0.105$&$1.129$&$9.905$&$362.942$\\
\hline
0.012500&$6.852e-01$&$4.096e-01$&$1.946e+00$&$2.718e+00$\\
&$3.577e-01$&$2.165e-01$&$1.129e+00$&$2.718e+00$\\
&$4.120e-01$&$2.255e-01$&$1.135e+00$&$2.722e+00$\\
&$0.164$&$1.302$&$23.119$&$385.953$\\
\hline
0.0062500&$6.947e-01$&$4.183e-01$&$2.312e-01$&$4.006e+00$\\
&$3.594e-01$&$2.179e-01$&$1.199e-01$&$5.667e-01$\\
&$4.146e-01$&$2.271e-01$&$1.212e-01$&$5.851e-01$\\
&$0.270$&$1.518$&$23.660$&$410.648$\\
\hline
0.0031250&$6.995e-01$&$4.236e-01$&$2.360e-01$&$1.224e-01$\\
&$3.603e-01$&$2.187e-01$&$1.206e-01$&$6.315e-02$\\
&$4.160e-01$&$2.281e-01$&$1.219e-01$&$6.333e-02$\\
&$0.487$&$2.468$&$31.381$&$473.422$\\
\hline
0.0015625&$7.019e-01$&$4.263e-01$&$2.384e-01$&$1.250e-01$\\
&$3.608e-01$&$2.191e-01$&$1.209e-01$&$6.349e-02$\\
&$4.167e-01$&$2.285e-01$&$1.223e-01$&$6.367e-02$\\
&$0.898$&$4.232$&$45.597$&$524.872$\\
\hline
\end{tabular}


\subsubsection{$\mu = 0.01, p(\rho) = 10\rho$}
$$\Vert g - \ln(\rho)\Vert$$
\begin{tabular}{*{5}{|c}|}
\hline
\diagbox{$\tau$}{$h$}&0.050000&0.025000&0.012500&0.0062500\\
\hline
0.050000&$4.316e+90$&$3.599e+00$&$3.527e+00$&$3.528e+00$\\
&$1.224e+90$&$3.364e+00$&$3.281e+00$&$3.281e+00$\\
&$2.111e+90$&$3.376e+00$&$3.282e+00$&$3.281e+00$\\
&$0.130$&$0.596$&$9.003$&$162.554$\\
\hline
0.025000&$1.459e-01$&$2.094e+23$&$1.257e+19$&$2.118e+13$\\
&$1.519e-01$&$5.950e+21$&$1.558e+18$&$3.891e+12$\\
&$1.611e-01$&$1.031e+22$&$2.106e+18$&$5.897e+12$\\
&$0.092$&$0.731$&$5.046$&$229.021$\\
\hline
0.012500&$1.556e-01$&$1.364e-01$&$1.068e+11$&$1.575e+10$\\
&$1.529e-01$&$7.938e-02$&$6.381e+09$&$1.376e+09$\\
&$1.626e-01$&$8.133e-02$&$9.996e+09$&$1.672e+09$\\
&$0.134$&$0.789$&$7.744$&$141.340$\\
\hline
0.0062500&$1.601e-01$&$1.476e-01$&$7.909e-02$&$1.657e+55$\\
&$1.534e-01$&$7.997e-02$&$4.163e-02$&$2.061e+54$\\
&$1.634e-01$&$8.203e-02$&$4.196e-02$&$2.765e+54$\\
&$0.235$&$1.200$&$10.541$&$217.226$\\
\hline
0.0031250&$1.623e-01$&$1.531e-01$&$8.570e-02$&$2.665e-02$\\
&$1.537e-01$&$8.031e-02$&$4.189e-02$&$2.121e-02$\\
&$1.638e-01$&$8.242e-02$&$4.223e-02$&$2.125e-02$\\
&$0.448$&$1.941$&$12.577$&$278.867$\\
\hline
0.0015625&$1.633e-01$&$1.559e-01$&$8.921e-02$&$2.836e-02$\\
&$1.539e-01$&$8.048e-02$&$4.203e-02$&$2.126e-02$\\
&$1.640e-01$&$8.262e-02$&$4.239e-02$&$2.130e-02$\\
&$0.828$&$3.438$&$21.066$&$310.699$\\
\hline
\end{tabular}
$$\Vert v_1 - u_1 \Vert$$
\begin{tabular}{*{5}{|c}|}
\hline
\diagbox{$\tau$}{$h$}&0.050000&0.025000&0.012500&0.0062500\\
\hline
0.050000&$3.679e-01$&$4.721e+00$&$2.968e+00$&$2.970e+00$\\
&$3.679e-01$&$2.870e+00$&$2.450e+00$&$2.453e+00$\\
&$4.023e-01$&$3.062e+00$&$2.484e+00$&$2.462e+00$\\
&$0.130$&$0.596$&$9.003$&$162.554$\\
\hline
0.025000&$1.134e+00$&$3.679e-01$&$3.679e-01$&$3.679e-01$\\
&$5.406e-01$&$3.679e-01$&$3.679e-01$&$3.679e-01$\\
&$6.785e-01$&$3.768e-01$&$3.701e-01$&$3.684e-01$\\
&$0.092$&$0.731$&$5.046$&$229.021$\\
\hline
0.012500&$1.152e+00$&$9.201e-01$&$3.679e-01$&$3.679e-01$\\
&$5.389e-01$&$3.398e-01$&$3.679e-01$&$3.679e-01$\\
&$6.811e-01$&$3.812e-01$&$3.701e-01$&$3.684e-01$\\
&$0.134$&$0.789$&$7.744$&$141.340$\\
\hline
0.0062500&$1.160e+00$&$9.442e-01$&$7.073e-01$&$3.679e-01$\\
&$5.381e-01$&$3.410e-01$&$2.004e-01$&$3.679e-01$\\
&$6.825e-01$&$3.838e-01$&$2.104e-01$&$3.684e-01$\\
&$0.235$&$1.200$&$10.541$&$217.226$\\
\hline
0.0031250&$1.164e+00$&$9.564e-01$&$7.283e-01$&$4.405e-01$\\
&$5.377e-01$&$3.416e-01$&$2.022e-01$&$1.101e-01$\\
&$6.832e-01$&$3.852e-01$&$2.126e-01$&$1.119e-01$\\
&$0.448$&$1.941$&$12.577$&$278.867$\\
\hline
0.0015625&$1.166e+00$&$9.625e-01$&$7.387e-01$&$4.530e-01$\\
&$5.375e-01$&$3.420e-01$&$2.032e-01$&$1.112e-01$\\
&$6.835e-01$&$3.859e-01$&$2.137e-01$&$1.131e-01$\\
&$0.828$&$3.438$&$21.066$&$310.699$\\
\hline
\end{tabular}
$$\Vert v_2 - u_2\Vert$$
\begin{tabular}{*{5}{|c}|}
\hline
\diagbox{$\tau$}{$h$}&0.050000&0.025000&0.012500&0.0062500\\
\hline
0.050000&$2.718e+00$&$3.094e+00$&$1.750e+00$&$1.753e+00$\\
&$2.718e+00$&$2.398e+00$&$1.350e+00$&$1.351e+00$\\
&$2.972e+00$&$2.571e+00$&$1.363e+00$&$1.354e+00$\\
&$0.130$&$0.596$&$9.003$&$162.554$\\
\hline
0.025000&$1.266e+00$&$2.718e+00$&$2.718e+00$&$2.718e+00$\\
&$6.026e-01$&$2.718e+00$&$2.718e+00$&$2.718e+00$\\
&$7.416e-01$&$2.784e+00$&$2.735e+00$&$2.722e+00$\\
&$0.092$&$0.731$&$5.046$&$229.021$\\
\hline
0.012500&$1.291e+00$&$1.269e+00$&$2.718e+00$&$2.718e+00$\\
&$6.063e-01$&$4.180e-01$&$2.718e+00$&$2.718e+00$\\
&$7.495e-01$&$4.547e-01$&$2.735e+00$&$2.722e+00$\\
&$0.134$&$0.789$&$7.744$&$141.340$\\
\hline
0.0062500&$1.301e+00$&$1.297e+00$&$9.542e-01$&$2.718e+00$\\
&$6.075e-01$&$4.233e-01$&$2.612e-01$&$2.718e+00$\\
&$7.526e-01$&$4.613e-01$&$2.686e-01$&$2.722e+00$\\
&$0.235$&$1.200$&$10.541$&$217.226$\\
\hline
0.0031250&$1.305e+00$&$1.310e+00$&$9.852e-01$&$5.713e-01$\\
&$6.079e-01$&$4.258e-01$&$2.662e-01$&$1.455e-01$\\
&$7.538e-01$&$4.645e-01$&$2.740e-01$&$1.466e-01$\\
&$0.448$&$1.941$&$12.577$&$278.867$\\
\hline
0.0015625&$1.307e+00$&$1.316e+00$&$1.001e+00$&$5.891e-01$\\
&$6.081e-01$&$4.270e-01$&$2.688e-01$&$1.482e-01$\\
&$7.544e-01$&$4.660e-01$&$2.767e-01$&$1.494e-01$\\
&$0.828$&$3.438$&$21.066$&$310.699$\\
\hline
\end{tabular}


\subsubsection{$\mu = 0.001, p(\rho) = 10\rho$}
$$\Vert g - \ln(\rho)\Vert$$
\begin{tabular}{*{5}{|c}|}
\hline
\diagbox{$\tau$}{$h$}&0.050000&0.025000&0.012500&0.0062500\\
\hline
0.050000&$2.208e+25$&$6.004e+09$&$6.300e+16$&$3.000e+00$\\
&$5.048e+24$&$4.308e+08$&$2.220e+15$&$3.266e+00$\\
&$7.992e+24$&$5.703e+08$&$4.090e+15$&$3.266e+00$\\
&$0.117$&$1.176$&$23.632$&$288.163$\\
\hline
0.025000&$1.806e-01$&$3.646e+00$&$1.521e+01$&$3.238e+01$\\
&$1.569e-01$&$3.069e+00$&$3.928e+00$&$7.948e+00$\\
&$1.684e-01$&$3.082e+00$&$4.201e+00$&$8.422e+00$\\
&$0.078$&$1.364$&$17.951$&$303.908$\\
\hline
0.012500&$1.896e-01$&$6.836e-01$&$6.064e+02$&$3.000e+00$\\
&$1.579e-01$&$3.502e-01$&$3.093e+01$&$3.266e+00$\\
&$1.698e-01$&$3.902e-01$&$4.977e+01$&$3.266e+00$\\
&$0.146$&$1.114$&$14.382$&$260.555$\\
\hline
0.0062500&$1.933e-01$&$2.094e-01$&$2.031e+00$&$2.552e+45$\\
&$1.584e-01$&$8.578e-02$&$1.149e+00$&$2.091e+44$\\
&$1.705e-01$&$8.905e-02$&$1.162e+00$&$2.158e+44$\\
&$0.245$&$1.146$&$11.297$&$287.550$\\
\hline
0.0031250&$1.946e-01$&$2.153e-01$&$1.658e-01$&$1.836e+00$\\
&$1.587e-01$&$8.623e-02$&$4.641e-02$&$1.411e+00$\\
&$1.709e-01$&$8.958e-02$&$4.718e-02$&$1.419e+00$\\
&$0.450$&$1.899$&$11.031$&$252.835$\\
\hline
0.0015625&$1.952e-01$&$2.182e-01$&$1.721e-01$&$7.224e-02$\\
&$1.589e-01$&$8.646e-02$&$4.678e-02$&$2.299e-02$\\
&$1.711e-01$&$8.984e-02$&$4.758e-02$&$2.309e-02$\\
&$0.836$&$3.422$&$18.290$&$241.715$\\
\hline
\end{tabular}
$$\Vert v_1 - u_1 \Vert$$
\begin{tabular}{*{5}{|c}|}
\hline
\diagbox{$\tau$}{$h$}&0.050000&0.025000&0.012500&0.0062500\\
\hline
0.050000&$0.000e+00$&$3.679e-01$&$3.679e-01$&$3.679e-01$\\
&$-nan$&$3.679e-01$&$3.679e-01$&$3.679e-01$\\
&$-nan$&$3.768e-01$&$3.701e-01$&$3.684e-01$\\
&$0.117$&$1.176$&$23.632$&$288.163$\\
\hline
0.025000&$1.275e+00$&$3.662e+00$&$1.061e+03$&$2.637e+06$\\
&$5.865e-01$&$2.446e+00$&$2.571e+01$&$3.023e+04$\\
&$7.567e-01$&$2.721e+00$&$4.601e+01$&$4.692e+04$\\
&$0.078$&$1.364$&$17.951$&$303.908$\\
\hline
0.012500&$1.300e+00$&$1.888e+00$&$0.000e+00$&$3.679e-01$\\
&$5.835e-01$&$1.109e+00$&$-nan$&$3.679e-01$\\
&$7.581e-01$&$1.261e+00$&$-nan$&$3.684e-01$\\
&$0.146$&$1.114$&$14.382$&$260.555$\\
\hline
0.0062500&$1.302e+00$&$1.123e+00$&$4.247e+00$&$3.679e-01$\\
&$5.818e-01$&$3.913e-01$&$2.241e+00$&$3.679e-01$\\
&$7.587e-01$&$4.552e-01$&$2.319e+00$&$3.684e-01$\\
&$0.245$&$1.146$&$11.297$&$287.550$\\
\hline
0.0031250&$1.303e+00$&$1.136e+00$&$9.438e-01$&$4.180e+00$\\
&$5.809e-01$&$3.920e-01$&$2.556e-01$&$2.439e+00$\\
&$7.590e-01$&$4.568e-01$&$2.758e-01$&$2.487e+00$\\
&$0.450$&$1.899$&$11.031$&$252.835$\\
\hline
0.0015625&$1.304e+00$&$1.142e+00$&$9.569e-01$&$6.951e-01$\\
&$5.804e-01$&$3.923e-01$&$2.573e-01$&$1.508e-01$\\
&$7.591e-01$&$4.576e-01$&$2.779e-01$&$1.558e-01$\\
&$0.836$&$3.422$&$18.290$&$241.715$\\
\hline
\end{tabular}
$$\Vert v_2 - u_2\Vert$$
\begin{tabular}{*{5}{|c}|}
\hline
\diagbox{$\tau$}{$h$}&0.050000&0.025000&0.012500&0.0062500\\
\hline
0.050000&$0.000e+00$&$2.718e+00$&$2.718e+00$&$2.718e+00$\\
&$-nan$&$2.718e+00$&$2.718e+00$&$2.718e+00$\\
&$-nan$&$2.784e+00$&$2.735e+00$&$2.722e+00$\\
&$0.117$&$1.176$&$23.632$&$288.163$\\
\hline
0.025000&$1.379e+00$&$4.017e+00$&$1.459e+03$&$6.445e+05$\\
&$6.573e-01$&$1.710e+00$&$3.264e+01$&$9.316e+03$\\
&$8.265e-01$&$1.949e+00$&$5.974e+01$&$1.405e+04$\\
&$0.078$&$1.364$&$17.951$&$303.908$\\
\hline
0.012500&$1.389e+00$&$2.737e+00$&$0.000e+00$&$2.718e+00$\\
&$6.574e-01$&$1.156e+00$&$-nan$&$2.718e+00$\\
&$8.300e-01$&$1.244e+00$&$-nan$&$2.722e+00$\\
&$0.146$&$1.114$&$14.382$&$260.555$\\
\hline
0.0062500&$1.391e+00$&$1.486e+00$&$5.093e+00$&$2.718e+00$\\
&$6.563e-01$&$4.888e-01$&$2.453e+00$&$2.718e+00$\\
&$8.303e-01$&$5.446e-01$&$2.507e+00$&$2.722e+00$\\
&$0.245$&$1.146$&$11.297$&$287.550$\\
\hline
0.0031250&$1.391e+00$&$1.495e+00$&$1.256e+00$&$5.532e+00$\\
&$6.556e-01$&$4.900e-01$&$3.433e-01$&$2.994e+00$\\
&$8.301e-01$&$5.465e-01$&$3.582e-01$&$3.021e+00$\\
&$0.450$&$1.899$&$11.031$&$252.835$\\
\hline
0.0015625&$1.391e+00$&$1.498e+00$&$1.274e+00$&$7.903e-01$\\
&$6.551e-01$&$4.905e-01$&$3.466e-01$&$2.037e-01$\\
&$8.299e-01$&$5.473e-01$&$3.618e-01$&$2.064e-01$\\
&$0.836$&$3.422$&$18.290$&$241.715$\\
\hline
\end{tabular}


\subsubsection{$\mu = 0.1, p(\rho) = 100\rho$}
$$\Vert g - \ln(\rho)\Vert$$
\begin{tabular}{*{5}{|c}|}
\hline
\diagbox{$\tau$}{$h$}&0.050000&0.025000&0.012500&0.0062500\\
\hline
0.050000&$5.564e+05$&$3.000e+00$&$3.000e+00$&$3.000e+00$\\
&$3.879e+05$&$3.267e+00$&$3.266e+00$&$3.266e+00$\\
&$4.257e+05$&$3.267e+00$&$3.266e+00$&$3.266e+00$\\
&$0.130$&$1.106$&$27.208$&$323.591$\\
\hline
0.025000&$5.270e+04$&$8.137e+03$&$1.896e+06$&$4.072e+14$\\
&$1.304e+04$&$2.681e+02$&$1.560e+06$&$6.254e+13$\\
&$1.692e+04$&$5.248e+02$&$1.590e+06$&$8.176e+13$\\
&$0.413$&$4.020$&$28.215$&$452.343$\\
\hline
0.012500&$3.499e+03$&$4.049e+11$&$3.449e+31$&$7.370e+12$\\
&$1.594e+02$&$1.875e+11$&$5.200e+30$&$9.108e+10$\\
&$2.756e+02$&$2.055e+11$&$8.819e+30$&$1.577e+11$\\
&$0.438$&$0.887$&$7.039$&$188.617$\\
\hline
0.0062500&$1.758e-02$&$1.377e+18$&$2.433e+05$&$2.718e+03$\\
&$1.809e-02$&$5.576e+17$&$2.377e+03$&$2.572e+01$\\
&$1.880e-02$&$7.163e+17$&$4.116e+03$&$4.351e+01$\\
&$0.327$&$1.977$&$83.407$&$610.232$\\
\hline
0.0031250&$1.752e-02$&$8.678e-03$&$2.805e+00$&$1.822e+10$\\
&$1.810e-02$&$9.071e-03$&$3.371e+00$&$3.873e+09$\\
&$1.881e-02$&$9.180e-03$&$3.373e+00$&$4.307e+09$\\
&$0.549$&$2.664$&$80.657$&$262.221$\\
\hline
0.0015625&$1.749e-02$&$8.674e-03$&$4.709e-03$&$2.010e+00$\\
&$1.810e-02$&$9.077e-03$&$4.683e-03$&$2.267e+00$\\
&$1.882e-02$&$9.186e-03$&$4.698e-03$&$2.267e+00$\\
&$0.925$&$4.691$&$60.003$&$615.415$\\
\hline
\end{tabular}
$$\Vert v_1 - u_1 \Vert$$
\begin{tabular}{*{5}{|c}|}
\hline
\diagbox{$\tau$}{$h$}&0.050000&0.025000&0.012500&0.0062500\\
\hline
0.050000&$0.000e+00$&$3.679e-01$&$3.679e-01$&$3.679e-01$\\
&$-nan$&$3.679e-01$&$3.679e-01$&$3.679e-01$\\
&$-nan$&$3.768e-01$&$3.701e-01$&$3.684e-01$\\
&$0.130$&$1.106$&$27.208$&$323.591$\\
\hline
0.025000&$2.266e+03$&$3.297e+00$&$3.679e-01$&$0.000e+00$\\
&$6.380e+02$&$2.437e+00$&$3.679e-01$&$-nan$\\
&$7.948e+02$&$2.460e+00$&$3.701e-01$&$-nan$\\
&$0.413$&$4.020$&$28.215$&$452.343$\\
\hline
0.012500&$3.131e+00$&$3.679e-01$&$3.679e-01$&$3.679e-01$\\
&$1.628e+00$&$3.679e-01$&$3.679e-01$&$3.679e-01$\\
&$1.817e+00$&$3.768e-01$&$3.701e-01$&$3.684e-01$\\
&$0.438$&$0.887$&$7.039$&$188.617$\\
\hline
0.0062500&$6.672e-01$&$3.679e-01$&$3.179e+00$&$2.569e+00$\\
&$3.868e-01$&$3.679e-01$&$1.597e+00$&$2.034e+00$\\
&$4.458e-01$&$3.768e-01$&$1.622e+00$&$2.041e+00$\\
&$0.327$&$1.977$&$83.407$&$610.232$\\
\hline
0.0031250&$6.679e-01$&$4.094e-01$&$6.230e+00$&$3.679e-01$\\
&$3.864e-01$&$2.181e-01$&$1.426e+00$&$3.679e-01$\\
&$4.456e-01$&$2.286e-01$&$1.522e+00$&$3.684e-01$\\
&$0.549$&$2.664$&$80.657$&$262.221$\\
\hline
0.0015625&$6.683e-01$&$4.096e-01$&$2.251e-01$&$4.108e+00$\\
&$3.862e-01$&$2.178e-01$&$1.167e-01$&$1.426e+00$\\
&$4.455e-01$&$2.284e-01$&$1.184e-01$&$1.456e+00$\\
&$0.925$&$4.691$&$60.003$&$615.415$\\
\hline
\end{tabular}
$$\Vert v_2 - u_2\Vert$$
\begin{tabular}{*{5}{|c}|}
\hline
\diagbox{$\tau$}{$h$}&0.050000&0.025000&0.012500&0.0062500\\
\hline
0.050000&$0.000e+00$&$2.718e+00$&$2.718e+00$&$2.718e+00$\\
&$-nan$&$2.718e+00$&$2.718e+00$&$2.718e+00$\\
&$-nan$&$2.784e+00$&$2.735e+00$&$2.722e+00$\\
&$0.130$&$1.106$&$27.208$&$323.591$\\
\hline
0.025000&$5.423e+01$&$2.696e+00$&$2.718e+00$&$0.000e+00$\\
&$1.690e+01$&$1.855e+00$&$2.718e+00$&$-nan$\\
&$2.027e+01$&$1.890e+00$&$2.735e+00$&$-nan$\\
&$0.413$&$4.020$&$28.215$&$452.343$\\
\hline
0.012500&$2.309e+00$&$2.718e+00$&$2.718e+00$&$2.718e+00$\\
&$1.153e+00$&$2.718e+00$&$2.718e+00$&$2.718e+00$\\
&$1.285e+00$&$2.784e+00$&$2.735e+00$&$2.722e+00$\\
&$0.438$&$0.887$&$7.039$&$188.617$\\
\hline
0.0062500&$6.632e-01$&$2.718e+00$&$3.030e+00$&$3.041e+00$\\
&$3.454e-01$&$2.718e+00$&$1.354e+00$&$1.286e+00$\\
&$3.999e-01$&$2.784e+00$&$1.383e+00$&$1.296e+00$\\
&$0.327$&$1.977$&$83.407$&$610.232$\\
\hline
0.0031250&$6.678e-01$&$4.150e-01$&$7.841e+00$&$2.718e+00$\\
&$3.462e-01$&$2.128e-01$&$1.279e+00$&$2.718e+00$\\
&$4.011e-01$&$2.222e-01$&$1.389e+00$&$2.722e+00$\\
&$0.549$&$2.664$&$80.657$&$262.221$\\
\hline
0.0015625&$6.702e-01$&$4.176e-01$&$2.348e-01$&$6.014e+00$\\
&$3.465e-01$&$2.132e-01$&$1.186e-01$&$1.078e+00$\\
&$4.017e-01$&$2.226e-01$&$1.199e-01$&$1.115e+00$\\
&$0.925$&$4.691$&$60.003$&$615.415$\\
\hline
\end{tabular}


\subsubsection{$\mu = 0.01, p(\rho) = 100\rho$}
$$\Vert g - \ln(\rho)\Vert$$
\begin{tabular}{*{5}{|c}|}
\hline
\diagbox{$\tau$}{$h$}&0.050000&0.025000&0.012500&0.0062500\\
\hline
0.050000&$3.507e+11$&$3.379e+01$&$2.277e+21$&$9.626e+23$\\
&$2.281e+11$&$2.396e+01$&$1.067e+21$&$4.221e+23$\\
&$2.621e+11$&$2.530e+01$&$1.109e+21$&$4.281e+23$\\
&$0.137$&$1.624$&$11.594$&$259.763$\\
\hline
0.025000&$3.411e+00$&$7.624e+38$&$2.577e+14$&$1.731e+16$\\
&$3.354e+00$&$1.822e+38$&$1.121e+14$&$7.398e+15$\\
&$3.393e+00$&$3.513e+38$&$1.175e+14$&$7.500e+15$\\
&$0.801$&$1.618$&$22.919$&$397.370$\\
\hline
0.012500&$2.950e+00$&$3.000e+00$&$3.012e+00$&$8.827e+06$\\
&$3.184e+00$&$3.267e+00$&$3.250e+00$&$4.582e+05$\\
&$3.209e+00$&$3.267e+00$&$3.251e+00$&$6.195e+05$\\
&$0.301$&$2.346$&$18.247$&$535.474$\\
\hline
0.0062500&$1.694e-02$&$4.729e+25$&$2.119e+12$&$7.979e+57$\\
&$1.701e-02$&$9.173e+23$&$1.606e+11$&$7.794e+55$\\
&$1.828e-02$&$1.604e+24$&$2.630e+11$&$1.598e+56$\\
&$0.256$&$1.250$&$12.007$&$251.566$\\
\hline
0.0031250&$1.713e-02$&$1.662e-02$&$6.623e+22$&$1.539e+58$\\
&$1.705e-02$&$8.584e-03$&$8.821e+21$&$8.976e+56$\\
&$1.832e-02$&$8.867e-03$&$9.666e+21$&$1.201e+57$\\
&$0.977$&$2.225$&$15.303$&$265.209$\\
\hline
0.0015625&$1.715e-02$&$1.692e-02$&$1.001e-02$&$2.100e+08$\\
&$1.706e-02$&$8.608e-03$&$4.472e-03$&$8.040e+06$\\
&$1.834e-02$&$8.893e-03$&$4.522e-03$&$9.797e+06$\\
&$0.905$&$4.405$&$24.970$&$245.172$\\
\hline
\end{tabular}
$$\Vert v_1 - u_1 \Vert$$
\begin{tabular}{*{5}{|c}|}
\hline
\diagbox{$\tau$}{$h$}&0.050000&0.025000&0.012500&0.0062500\\
\hline
0.050000&$0.000e+00$&$2.855e+17$&$0.000e+00$&$0.000e+00$\\
&$-nan$&$3.312e+16$&$-nan$&$-nan$\\
&$-nan$&$3.870e+16$&$-nan$&$-nan$\\
&$0.137$&$1.624$&$11.594$&$259.763$\\
\hline
0.025000&$4.000e+01$&$3.679e-01$&$0.000e+00$&$0.000e+00$\\
&$1.045e+01$&$3.679e-01$&$-nan$&$-nan$\\
&$1.352e+01$&$3.768e-01$&$-nan$&$-nan$\\
&$0.801$&$1.618$&$22.919$&$397.370$\\
\hline
0.012500&$1.359e+01$&$3.679e-01$&$2.902e+00$&$8.236e+09$\\
&$7.872e+00$&$3.679e-01$&$2.909e+00$&$2.192e+08$\\
&$8.545e+00$&$3.768e-01$&$2.943e+00$&$2.556e+08$\\
&$0.301$&$2.346$&$18.247$&$535.474$\\
\hline
0.0062500&$1.172e+00$&$3.679e-01$&$3.679e-01$&$3.679e-01$\\
&$5.730e-01$&$3.679e-01$&$3.679e-01$&$3.679e-01$\\
&$7.164e-01$&$3.768e-01$&$3.701e-01$&$3.684e-01$\\
&$0.256$&$1.250$&$12.007$&$251.566$\\
\hline
0.0031250&$1.174e+00$&$9.934e-01$&$3.679e-01$&$3.679e-01$\\
&$5.719e-01$&$3.625e-01$&$3.679e-01$&$3.679e-01$\\
&$7.162e-01$&$4.047e-01$&$3.701e-01$&$3.684e-01$\\
&$0.977$&$2.225$&$15.303$&$265.209$\\
\hline
0.0015625&$1.174e+00$&$1.001e+00$&$7.635e-01$&$3.679e-01$\\
&$5.714e-01$&$3.629e-01$&$2.143e-01$&$3.679e-01$\\
&$7.161e-01$&$4.054e-01$&$2.243e-01$&$3.684e-01$\\
&$0.905$&$4.405$&$24.970$&$245.172$\\
\hline
\end{tabular}
$$\Vert v_2 - u_2\Vert$$
\begin{tabular}{*{5}{|c}|}
\hline
\diagbox{$\tau$}{$h$}&0.050000&0.025000&0.012500&0.0062500\\
\hline
0.050000&$0.000e+00$&$1.054e+08$&$0.000e+00$&$0.000e+00$\\
&$-nan$&$1.949e+07$&$-nan$&$-nan$\\
&$-nan$&$2.240e+07$&$-nan$&$-nan$\\
&$0.137$&$1.624$&$11.594$&$259.763$\\
\hline
0.025000&$3.311e+01$&$2.718e+00$&$0.000e+00$&$0.000e+00$\\
&$1.046e+01$&$2.718e+00$&$-nan$&$-nan$\\
&$1.384e+01$&$2.784e+00$&$-nan$&$-nan$\\
&$0.801$&$1.618$&$22.919$&$397.370$\\
\hline
0.012500&$1.256e+01$&$2.718e+00$&$2.513e+00$&$6.106e+43$\\
&$7.582e+00$&$2.718e+00$&$2.681e+00$&$1.132e+42$\\
&$8.864e+00$&$2.784e+00$&$2.701e+00$&$1.478e+42$\\
&$0.301$&$2.346$&$18.247$&$535.474$\\
\hline
0.0062500&$1.328e+00$&$2.718e+00$&$2.718e+00$&$2.718e+00$\\
&$5.700e-01$&$2.718e+00$&$2.718e+00$&$2.718e+00$\\
&$7.140e-01$&$2.784e+00$&$2.735e+00$&$2.722e+00$\\
&$0.256$&$1.250$&$12.007$&$251.566$\\
\hline
0.0031250&$1.332e+00$&$1.255e+00$&$2.718e+00$&$2.718e+00$\\
&$5.702e-01$&$4.079e-01$&$2.718e+00$&$2.718e+00$\\
&$7.150e-01$&$4.472e-01$&$2.735e+00$&$2.722e+00$\\
&$0.977$&$2.225$&$15.303$&$265.209$\\
\hline
0.0015625&$1.334e+00$&$1.263e+00$&$1.037e+00$&$2.718e+00$\\
&$5.702e-01$&$4.087e-01$&$2.636e-01$&$2.718e+00$\\
&$7.154e-01$&$4.484e-01$&$2.718e-01$&$2.722e+00$\\
&$0.905$&$4.405$&$24.970$&$245.172$\\
\hline
\end{tabular}


\subsubsection{$\mu = 0.001, p(\rho) = 100\rho$}
$$\Vert g - \ln(\rho)\Vert$$
\begin{tabular}{*{5}{|c}|}
\hline
\diagbox{$\tau$}{$h$}&0.050000&0.025000&0.012500&0.0062500\\
\hline
0.050000&$1.903e+14$&$5.326e+01$&$7.837e+40$&$2.795e+56$\\
&$1.195e+14$&$2.832e+01$&$3.230e+40$&$1.011e+56$\\
&$1.386e+14$&$3.063e+01$&$3.428e+40$&$1.039e+56$\\
&$0.078$&$1.296$&$18.353$&$301.584$\\
\hline
0.025000&$2.807e+09$&$3.020e+04$&$1.522e+31$&$3.012e+57$\\
&$1.707e+08$&$3.764e+03$&$6.005e+30$&$3.901e+56$\\
&$2.973e+08$&$4.431e+03$&$6.438e+30$&$4.488e+56$\\
&$0.247$&$2.298$&$76.290$&$368.078$\\
\hline
0.012500&$7.944e+11$&$1.081e+01$&$3.494e+00$&$2.937e+24$\\
&$1.661e+11$&$4.199e+00$&$3.251e+00$&$2.324e+23$\\
&$2.304e+11$&$4.793e+00$&$3.265e+00$&$3.607e+23$\\
&$0.199$&$3.229$&$22.489$&$383.919$\\
\hline
0.0062500&$1.819e-02$&$1.407e+10$&$2.988e+00$&$2.723e+01$\\
&$1.719e-02$&$1.333e+09$&$3.241e+00$&$5.833e+00$\\
&$1.870e-02$&$1.901e+09$&$3.242e+00$&$6.537e+00$\\
&$0.249$&$1.658$&$25.300$&$356.789$\\
\hline
0.0031250&$1.838e-02$&$2.326e-02$&$2.997e+00$&$8.082e+22$\\
&$1.723e-02$&$8.802e-03$&$3.261e+00$&$4.266e+20$\\
&$1.875e-02$&$9.252e-03$&$3.261e+00$&$8.154e+20$\\
&$0.445$&$1.953$&$16.293$&$274.381$\\
\hline
0.0015625&$1.845e-02$&$2.353e-02$&$1.980e-02$&$2.933e+127$\\
&$1.724e-02$&$8.829e-03$&$4.767e-03$&$2.203e+125$\\
&$1.876e-02$&$9.282e-03$&$4.886e-03$&$4.623e+125$\\
&$0.840$&$3.451$&$20.167$&$298.578$\\
\hline
\end{tabular}
$$\Vert v_1 - u_1 \Vert$$
\begin{tabular}{*{5}{|c}|}
\hline
\diagbox{$\tau$}{$h$}&0.050000&0.025000&0.012500&0.0062500\\
\hline
0.050000&$0.000e+00$&$9.703e+25$&$0.000e+00$&$0.000e+00$\\
&$-nan$&$1.034e+25$&$-nan$&$-nan$\\
&$-nan$&$1.227e+25$&$-nan$&$-nan$\\
&$0.078$&$1.296$&$18.353$&$301.584$\\
\hline
0.025000&$3.679e-01$&$2.822e+20$&$0.000e+00$&$3.679e-01$\\
&$3.679e-01$&$8.452e+18$&$-nan$&$3.679e-01$\\
&$4.023e-01$&$1.483e+19$&$-nan$&$3.684e-01$\\
&$0.247$&$2.298$&$76.290$&$368.078$\\
\hline
0.012500&$3.679e-01$&$4.745e+04$&$1.113e+02$&$1.380e+18$\\
&$3.679e-01$&$4.668e+03$&$1.225e+01$&$1.356e+17$\\
&$4.023e-01$&$8.133e+03$&$1.809e+01$&$1.447e+17$\\
&$0.199$&$3.229$&$22.489$&$383.919$\\
\hline
0.0062500&$1.256e+00$&$3.679e-01$&$7.081e+00$&$6.439e+05$\\
&$6.115e-01$&$3.679e-01$&$2.875e+00$&$3.260e+04$\\
&$7.848e-01$&$3.768e-01$&$3.182e+00$&$5.476e+04$\\
&$0.249$&$1.658$&$25.300$&$356.789$\\
\hline
0.0031250&$1.256e+00$&$1.147e+00$&$6.678e-01$&$3.679e-01$\\
&$6.098e-01$&$4.086e-01$&$7.216e-01$&$3.679e-01$\\
&$7.839e-01$&$4.715e-01$&$7.312e-01$&$3.684e-01$\\
&$0.445$&$1.953$&$16.293$&$274.381$\\
\hline
0.0015625&$1.255e+00$&$1.153e+00$&$1.015e+00$&$3.679e-01$\\
&$6.089e-01$&$4.089e-01$&$2.667e-01$&$3.679e-01$\\
&$7.835e-01$&$4.722e-01$&$2.864e-01$&$3.684e-01$\\
&$0.840$&$3.451$&$20.167$&$298.578$\\
\hline
\end{tabular}
$$\Vert v_2 - u_2\Vert$$
\begin{tabular}{*{5}{|c}|}
\hline
\diagbox{$\tau$}{$h$}&0.050000&0.025000&0.012500&0.0062500\\
\hline
0.050000&$0.000e+00$&$1.686e+12$&$0.000e+00$&$0.000e+00$\\
&$-nan$&$2.430e+11$&$-nan$&$-nan$\\
&$-nan$&$2.898e+11$&$-nan$&$-nan$\\
&$0.078$&$1.296$&$18.353$&$301.584$\\
\hline
0.025000&$2.718e+00$&$2.661e+20$&$0.000e+00$&$2.718e+00$\\
&$2.718e+00$&$9.953e+18$&$-nan$&$2.718e+00$\\
&$2.972e+00$&$1.636e+19$&$-nan$&$2.722e+00$\\
&$0.247$&$2.298$&$76.290$&$368.078$\\
\hline
0.012500&$2.718e+00$&$1.803e+03$&$4.463e+01$&$0.000e+00$\\
&$2.718e+00$&$1.224e+02$&$1.069e+01$&$-nan$\\
&$2.972e+00$&$2.207e+02$&$1.459e+01$&$-nan$\\
&$0.199$&$3.229$&$22.489$&$383.919$\\
\hline
0.0062500&$1.410e+00$&$2.718e+00$&$5.855e+00$&$1.833e+06$\\
&$6.135e-01$&$2.718e+00$&$3.178e+00$&$2.690e+04$\\
&$7.870e-01$&$2.784e+00$&$3.542e+00$&$4.469e+04$\\
&$0.249$&$1.658$&$25.300$&$356.789$\\
\hline
0.0031250&$1.410e+00$&$1.445e+00$&$2.922e+00$&$2.718e+00$\\
&$6.125e-01$&$4.659e-01$&$2.679e+00$&$2.718e+00$\\
&$7.863e-01$&$5.231e-01$&$2.697e+00$&$2.722e+00$\\
&$0.445$&$1.953$&$16.293$&$274.381$\\
\hline
0.0015625&$1.410e+00$&$1.452e+00$&$1.297e+00$&$2.718e+00$\\
&$6.119e-01$&$4.658e-01$&$3.384e-01$&$2.718e+00$\\
&$7.859e-01$&$5.233e-01$&$3.541e-01$&$2.722e+00$\\
&$0.840$&$3.451$&$20.167$&$298.578$\\
\hline
\end{tabular}


\subsubsection{$\mu = 0.1, p(\rho) = \rho^{1.4}$}
$$\Vert g - \ln(\rho)\Vert$$
\begin{tabular}{*{5}{|c}|}
\hline
\diagbox{$\tau$}{$h$}&0.050000&0.025000&0.012500&0.0062500\\
\hline
0.050000&$3.865e-01$&$8.855e+05$&$2.027e+06$&$6.570e+02$\\
&$4.027e-01$&$4.667e+04$&$1.600e+05$&$1.794e+01$\\
&$4.166e-01$&$5.658e+04$&$2.111e+05$&$2.633e+01$\\
&$0.063$&$0.725$&$16.538$&$294.460$\\
\hline
0.025000&$4.561e-01$&$1.893e-01$&$3.172e+00$&$8.171e+04$\\
&$3.887e-01$&$2.076e-01$&$3.248e+00$&$1.820e+03$\\
&$4.052e-01$&$2.098e-01$&$3.249e+00$&$2.335e+03$\\
&$0.104$&$0.752$&$12.356$&$206.069$\\
\hline
0.012500&$5.022e-01$&$1.942e-01$&$1.164e-01$&$3.024e+01$\\
&$3.824e-01$&$1.996e-01$&$1.079e-01$&$2.684e-01$\\
&$4.006e-01$&$2.021e-01$&$1.083e-01$&$3.966e-01$\\
&$0.165$&$1.133$&$20.144$&$364.632$\\
\hline
0.0062500&$5.264e-01$&$2.105e-01$&$1.042e-01$&$7.507e-02$\\
&$3.795e-01$&$1.959e-01$&$1.035e-01$&$5.537e-02$\\
&$3.987e-01$&$1.986e-01$&$1.039e-01$&$5.542e-02$\\
&$0.282$&$1.573$&$24.190$&$419.530$\\
\hline
0.0031250&$5.388e-01$&$2.202e-01$&$1.066e-01$&$6.682e-02$\\
&$3.781e-01$&$1.942e-01$&$1.015e-01$&$5.303e-02$\\
&$3.978e-01$&$1.970e-01$&$1.019e-01$&$5.308e-02$\\
&$0.531$&$2.628$&$26.155$&$465.171$\\
\hline
0.0015625&$5.451e-01$&$2.251e-01$&$1.078e-01$&$6.267e-02$\\
&$3.774e-01$&$1.934e-01$&$1.005e-01$&$5.195e-02$\\
&$3.974e-01$&$1.962e-01$&$1.009e-01$&$5.200e-02$\\
&$0.959$&$4.573$&$65.043$&$469.867$\\
\hline
\end{tabular}
$$\Vert v_1 - u_1 \Vert$$
\begin{tabular}{*{5}{|c}|}
\hline
\diagbox{$\tau$}{$h$}&0.050000&0.025000&0.012500&0.0062500\\
\hline
0.050000&$5.346e-01$&$2.463e+00$&$1.009e+04$&$7.261e+08$\\
&$3.272e-01$&$1.278e+00$&$6.274e+02$&$7.294e+06$\\
&$3.699e-01$&$1.298e+00$&$7.533e+02$&$1.051e+07$\\
&$0.063$&$0.725$&$16.538$&$294.460$\\
\hline
0.025000&$5.324e-01$&$3.165e-01$&$4.623e-01$&$6.602e+00$\\
&$3.107e-01$&$1.794e-01$&$3.744e-01$&$8.391e-01$\\
&$3.546e-01$&$1.877e-01$&$3.767e-01$&$8.572e-01$\\
&$0.104$&$0.752$&$12.356$&$206.069$\\
\hline
0.012500&$5.368e-01$&$3.098e-01$&$1.909e-01$&$1.491e+00$\\
&$3.033e-01$&$1.701e-01$&$9.550e-02$&$1.335e-01$\\
&$3.482e-01$&$1.788e-01$&$9.687e-02$&$1.428e-01$\\
&$0.165$&$1.133$&$20.144$&$364.632$\\
\hline
0.0062500&$5.391e-01$&$3.084e-01$&$1.866e-01$&$1.054e-01$\\
&$2.999e-01$&$1.661e-01$&$9.052e-02$&$4.956e-02$\\
&$3.454e-01$&$1.751e-01$&$9.198e-02$&$4.976e-02$\\
&$0.282$&$1.573$&$24.190$&$419.530$\\
\hline
0.0031250&$5.402e-01$&$3.089e-01$&$1.848e-01$&$1.030e-01$\\
&$2.983e-01$&$1.643e-01$&$8.838e-02$&$4.697e-02$\\
&$3.440e-01$&$1.734e-01$&$8.988e-02$&$4.719e-02$\\
&$0.531$&$2.628$&$26.155$&$465.171$\\
\hline
0.0015625&$5.408e-01$&$3.092e-01$&$1.840e-01$&$1.019e-01$\\
&$2.975e-01$&$1.634e-01$&$8.741e-02$&$4.587e-02$\\
&$3.434e-01$&$1.726e-01$&$8.893e-02$&$4.608e-02$\\
&$0.959$&$4.573$&$65.043$&$469.867$\\
\hline
\end{tabular}
$$\Vert v_2 - u_2\Vert$$
\begin{tabular}{*{5}{|c}|}
\hline
\diagbox{$\tau$}{$h$}&0.050000&0.025000&0.012500&0.0062500\\
\hline
0.050000&$6.945e-01$&$3.152e+00$&$2.767e+00$&$3.343e+00$\\
&$3.781e-01$&$1.330e+00$&$1.061e+00$&$1.569e+00$\\
&$4.331e-01$&$1.365e+00$&$1.077e+00$&$1.573e+00$\\
&$0.063$&$0.725$&$16.538$&$294.460$\\
\hline
0.025000&$7.371e-01$&$4.301e-01$&$2.660e+00$&$1.277e+01$\\
&$3.865e-01$&$2.327e-01$&$2.446e+00$&$5.946e-01$\\
&$4.446e-01$&$2.421e-01$&$2.461e+00$&$6.249e-01$\\
&$0.104$&$0.752$&$12.356$&$206.069$\\
\hline
0.012500&$7.587e-01$&$4.479e-01$&$2.433e-01$&$1.413e+00$\\
&$3.922e-01$&$2.369e-01$&$1.295e-01$&$1.761e-01$\\
&$4.521e-01$&$2.468e-01$&$1.309e-01$&$1.851e-01$\\
&$0.165$&$1.133$&$20.144$&$364.632$\\
\hline
0.0062500&$7.694e-01$&$4.568e-01$&$2.505e-01$&$1.302e-01$\\
&$3.954e-01$&$2.397e-01$&$1.318e-01$&$6.846e-02$\\
&$4.563e-01$&$2.498e-01$&$1.332e-01$&$6.865e-02$\\
&$0.282$&$1.573$&$24.190$&$419.530$\\
\hline
0.0031250&$7.748e-01$&$4.612e-01$&$2.542e-01$&$1.334e-01$\\
&$3.970e-01$&$2.413e-01$&$1.332e-01$&$6.966e-02$\\
&$4.585e-01$&$2.516e-01$&$1.347e-01$&$6.985e-02$\\
&$0.531$&$2.628$&$26.155$&$465.171$\\
\hline
0.0015625&$7.775e-01$&$4.634e-01$&$2.560e-01$&$1.350e-01$\\
&$3.979e-01$&$2.422e-01$&$1.341e-01$&$7.042e-02$\\
&$4.596e-01$&$2.525e-01$&$1.356e-01$&$7.062e-02$\\
&$0.959$&$4.573$&$65.043$&$469.867$\\
\hline
\end{tabular}


\subsubsection{$\mu = 0.01, p(\rho) = \rho^{1.4}$}
$$\Vert g - \ln(\rho)\Vert$$
\begin{tabular}{*{5}{|c}|}
\hline
\diagbox{$\tau$}{$h$}&0.050000&0.025000&0.012500&0.0062500\\
\hline
0.050000&$7.653e-01$&$4.457e+36$&$2.042e+20$&$1.071e+13$\\
&$5.034e-01$&$1.198e+36$&$1.892e+19$&$1.487e+12$\\
&$5.380e-01$&$2.305e+36$&$3.198e+19$&$2.072e+12$\\
&$0.071$&$1.505$&$8.072$&$246.701$\\
\hline
0.025000&$8.862e-01$&$4.450e-01$&$1.035e+96$&$3.994e+00$\\
&$4.905e-01$&$2.590e-01$&$1.538e+95$&$3.329e+00$\\
&$5.322e-01$&$2.684e-01$&$2.672e+95$&$3.330e+00$\\
&$0.104$&$0.619$&$8.642$&$174.074$\\
\hline
0.012500&$9.530e-01$&$5.155e-01$&$4.295e+00$&$1.865e+11$\\
&$4.858e-01$&$2.533e-01$&$3.371e+00$&$7.870e+09$\\
&$5.316e-01$&$2.628e-01$&$3.376e+00$&$1.394e+10$\\
&$0.147$&$0.882$&$8.926$&$224.441$\\
\hline
0.0062500&$9.882e-01$&$5.544e-01$&$2.784e-01$&$4.483e+00$\\
&$4.839e-01$&$2.518e-01$&$1.332e-01$&$3.421e+00$\\
&$5.319e-01$&$2.622e-01$&$1.349e-01$&$3.423e+00$\\
&$0.285$&$1.430$&$10.705$&$296.563$\\
\hline
0.0031250&$1.006e+00$&$5.746e-01$&$2.973e-01$&$1.455e-01$\\
&$4.831e-01$&$2.513e-01$&$1.323e-01$&$6.851e-02$\\
&$5.322e-01$&$2.622e-01$&$1.342e-01$&$6.877e-02$\\
&$0.501$&$2.454$&$14.204$&$306.916$\\
\hline
0.0015625&$1.015e+00$&$5.845e-01$&$3.072e-01$&$1.489e-01$\\
&$4.827e-01$&$2.512e-01$&$1.320e-01$&$6.782e-02$\\
&$5.324e-01$&$2.623e-01$&$1.339e-01$&$6.810e-02$\\
&$0.981$&$4.363$&$25.783$&$280.353$\\
\hline
\end{tabular}
$$\Vert v_1 - u_1 \Vert$$
\begin{tabular}{*{5}{|c}|}
\hline
\diagbox{$\tau$}{$h$}&0.050000&0.025000&0.012500&0.0062500\\
\hline
0.050000&$1.042e+00$&$3.679e-01$&$3.679e-01$&$3.679e-01$\\
&$5.079e-01$&$3.679e-01$&$3.679e-01$&$3.679e-01$\\
&$6.334e-01$&$3.768e-01$&$3.701e-01$&$3.684e-01$\\
&$0.071$&$1.505$&$8.072$&$246.701$\\
\hline
0.025000&$1.076e+00$&$7.617e-01$&$3.679e-01$&$7.971e-01$\\
&$5.042e-01$&$3.195e-01$&$3.679e-01$&$6.846e-01$\\
&$6.358e-01$&$3.620e-01$&$3.701e-01$&$6.863e-01$\\
&$0.104$&$0.619$&$8.642$&$174.074$\\
\hline
0.012500&$1.096e+00$&$7.841e-01$&$8.925e-01$&$3.679e-01$\\
&$5.032e-01$&$3.198e-01$&$7.527e-01$&$3.679e-01$\\
&$6.382e-01$&$3.613e-01$&$7.604e-01$&$3.684e-01$\\
&$0.147$&$0.882$&$8.926$&$224.441$\\
\hline
0.0062500&$1.108e+00$&$7.940e-01$&$6.272e-01$&$9.896e-01$\\
&$5.029e-01$&$3.213e-01$&$1.919e-01$&$8.065e-01$\\
&$6.398e-01$&$3.640e-01$&$2.025e-01$&$8.088e-01$\\
&$0.285$&$1.430$&$10.705$&$296.563$\\
\hline
0.0031250&$1.113e+00$&$7.987e-01$&$6.388e-01$&$4.267e-01$\\
&$5.028e-01$&$3.222e-01$&$1.934e-01$&$1.069e-01$\\
&$6.407e-01$&$3.656e-01$&$2.043e-01$&$1.089e-01$\\
&$0.501$&$2.454$&$14.204$&$306.916$\\
\hline
0.0015625&$1.117e+00$&$8.010e-01$&$6.457e-01$&$4.367e-01$\\
&$5.028e-01$&$3.227e-01$&$1.943e-01$&$1.078e-01$\\
&$6.411e-01$&$3.664e-01$&$2.053e-01$&$1.098e-01$\\
&$0.981$&$4.363$&$25.783$&$280.353$\\
\hline
\end{tabular}
$$\Vert v_2 - u_2\Vert$$
\begin{tabular}{*{5}{|c}|}
\hline
\diagbox{$\tau$}{$h$}&0.050000&0.025000&0.012500&0.0062500\\
\hline
0.050000&$1.396e+00$&$2.718e+00$&$2.718e+00$&$2.718e+00$\\
&$7.083e-01$&$2.718e+00$&$2.718e+00$&$2.718e+00$\\
&$8.480e-01$&$2.784e+00$&$2.735e+00$&$2.722e+00$\\
&$0.071$&$1.505$&$8.072$&$246.701$\\
\hline
0.025000&$1.504e+00$&$1.215e+00$&$2.718e+00$&$3.100e+00$\\
&$7.279e-01$&$4.879e-01$&$2.718e+00$&$2.204e+00$\\
&$8.765e-01$&$5.234e-01$&$2.735e+00$&$2.209e+00$\\
&$0.104$&$0.619$&$8.642$&$174.074$\\
\hline
0.012500&$1.556e+00$&$1.322e+00$&$3.126e+00$&$2.718e+00$\\
&$7.377e-01$&$5.027e-01$&$2.116e+00$&$2.718e+00$\\
&$8.906e-01$&$5.386e-01$&$2.134e+00$&$2.722e+00$\\
&$0.147$&$0.882$&$8.926$&$224.441$\\
\hline
0.0062500&$1.581e+00$&$1.380e+00$&$9.852e-01$&$3.178e+00$\\
&$7.425e-01$&$5.112e-01$&$3.069e-01$&$2.049e+00$\\
&$8.974e-01$&$5.483e-01$&$3.139e-01$&$2.053e+00$\\
&$0.285$&$1.430$&$10.705$&$296.563$\\
\hline
0.0031250&$1.593e+00$&$1.409e+00$&$1.025e+00$&$6.068e-01$\\
&$7.448e-01$&$5.156e-01$&$3.129e-01$&$1.688e-01$\\
&$9.006e-01$&$5.533e-01$&$3.201e-01$&$1.699e-01$\\
&$0.501$&$2.454$&$14.204$&$306.916$\\
\hline
0.0015625&$1.599e+00$&$1.423e+00$&$1.046e+00$&$6.279e-01$\\
&$7.459e-01$&$5.178e-01$&$3.161e-01$&$1.718e-01$\\
&$9.022e-01$&$5.558e-01$&$3.234e-01$&$1.729e-01$\\
&$0.981$&$4.363$&$25.783$&$280.353$\\
\hline
\end{tabular}


\subsubsection{$\mu = 0.001, p(\rho) = \rho^{1.4}$}
$$\Vert g - \ln(\rho)\Vert$$
\begin{tabular}{*{5}{|c}|}
\hline
\diagbox{$\tau$}{$h$}&0.050000&0.025000&0.012500&0.0062500\\
\hline
0.050000&$9.428e-01$&$7.162e+00$&$3.288e+04$&$1.376e+06$\\
&$5.354e-01$&$2.929e+00$&$2.771e+03$&$1.430e+05$\\
&$5.829e-01$&$3.076e+00$&$3.939e+03$&$1.673e+05$\\
&$0.059$&$0.396$&$12.054$&$211.687$\\
\hline
0.025000&$1.095e+00$&$6.588e-01$&$6.610e+30$&$2.056e+26$\\
&$5.273e-01$&$2.924e-01$&$7.170e+29$&$1.092e+25$\\
&$5.833e-01$&$3.225e-01$&$9.913e+29$&$1.203e+25$\\
&$0.095$&$0.534$&$13.683$&$210.557$\\
\hline
0.012500&$1.181e+00$&$7.771e-01$&$3.828e+37$&$2.976e+47$\\
&$5.246e-01$&$2.819e-01$&$4.879e+36$&$2.335e+45$\\
&$5.852e-01$&$2.984e-01$&$9.088e+36$&$4.088e+45$\\
&$0.159$&$0.911$&$10.242$&$227.025$\\
\hline
0.0062500&$1.226e+00$&$8.491e-01$&$4.943e-01$&$4.225e+19$\\
&$5.236e-01$&$2.835e-01$&$1.525e-01$&$9.374e+17$\\
&$5.865e-01$&$3.014e-01$&$1.561e-01$&$1.203e+18$\\
&$0.253$&$1.229$&$12.632$&$202.192$\\
\hline
0.0031250&$1.249e+00$&$8.856e-01$&$5.277e-01$&$2.189e-01$\\
&$5.232e-01$&$2.846e-01$&$1.539e-01$&$7.741e-02$\\
&$5.872e-01$&$3.031e-01$&$1.578e-01$&$7.794e-02$\\
&$0.488$&$2.180$&$12.696$&$242.456$\\
\hline
0.0015625&$1.260e+00$&$9.036e-01$&$5.444e-01$&$2.324e-01$\\
&$5.230e-01$&$2.852e-01$&$1.548e-01$&$7.759e-02$\\
&$5.876e-01$&$3.040e-01$&$1.589e-01$&$7.815e-02$\\
&$1.027$&$3.962$&$26.062$&$261.074$\\
\hline
\end{tabular}
$$\Vert v_1 - u_1 \Vert$$
\begin{tabular}{*{5}{|c}|}
\hline
\diagbox{$\tau$}{$h$}&0.050000&0.025000&0.012500&0.0062500\\
\hline
0.050000&$1.156e+00$&$4.566e+01$&$0.000e+00$&$0.000e+00$\\
&$5.611e-01$&$2.753e+00$&$-nan$&$-nan$\\
&$7.152e-01$&$4.495e+00$&$-nan$&$-nan$\\
&$0.059$&$0.396$&$12.054$&$211.687$\\
\hline
0.025000&$1.198e+00$&$9.942e-01$&$3.679e-01$&$3.679e-01$\\
&$5.592e-01$&$3.918e-01$&$3.679e-01$&$3.679e-01$\\
&$7.202e-01$&$4.811e-01$&$3.701e-01$&$3.684e-01$\\
&$0.095$&$0.534$&$13.683$&$210.557$\\
\hline
0.012500&$1.225e+00$&$9.120e-01$&$3.679e-01$&$3.679e-01$\\
&$5.588e-01$&$3.769e-01$&$3.679e-01$&$3.679e-01$\\
&$7.240e-01$&$4.366e-01$&$3.701e-01$&$3.684e-01$\\
&$0.159$&$0.911$&$10.242$&$227.025$\\
\hline
0.0062500&$1.241e+00$&$9.308e-01$&$7.952e-01$&$3.679e-01$\\
&$5.586e-01$&$3.793e-01$&$2.446e-01$&$3.679e-01$\\
&$7.261e-01$&$4.407e-01$&$2.645e-01$&$3.684e-01$\\
&$0.253$&$1.229$&$12.632$&$202.192$\\
\hline
0.0031250&$1.249e+00$&$9.392e-01$&$8.091e-01$&$6.329e-01$\\
&$5.586e-01$&$3.806e-01$&$2.472e-01$&$1.463e-01$\\
&$7.272e-01$&$4.429e-01$&$2.675e-01$&$1.516e-01$\\
&$0.488$&$2.180$&$12.696$&$242.456$\\
\hline
0.0015625&$1.253e+00$&$9.432e-01$&$8.160e-01$&$6.444e-01$\\
&$5.585e-01$&$3.813e-01$&$2.486e-01$&$1.479e-01$\\
&$7.278e-01$&$4.440e-01$&$2.692e-01$&$1.533e-01$\\
&$1.027$&$3.962$&$26.062$&$261.074$\\
\hline
\end{tabular}
$$\Vert v_2 - u_2\Vert$$
\begin{tabular}{*{5}{|c}|}
\hline
\diagbox{$\tau$}{$h$}&0.050000&0.025000&0.012500&0.0062500\\
\hline
0.050000&$1.591e+00$&$1.942e+02$&$1.355e+43$&$2.698e+10$\\
&$7.921e-01$&$5.452e+00$&$2.893e+41$&$3.893e+08$\\
&$9.615e-01$&$1.141e+01$&$5.238e+41$&$5.221e+08$\\
&$0.059$&$0.396$&$12.054$&$211.687$\\
\hline
0.025000&$1.736e+00$&$1.537e+00$&$2.718e+00$&$2.718e+00$\\
&$8.078e-01$&$5.909e-01$&$2.718e+00$&$2.718e+00$\\
&$9.858e-01$&$6.596e-01$&$2.735e+00$&$2.722e+00$\\
&$0.095$&$0.534$&$13.683$&$210.557$\\
\hline
0.012500&$1.792e+00$&$1.667e+00$&$2.718e+00$&$2.718e+00$\\
&$8.136e-01$&$5.923e-01$&$2.718e+00$&$2.718e+00$\\
&$9.949e-01$&$6.446e-01$&$2.735e+00$&$2.722e+00$\\
&$0.159$&$0.911$&$10.242$&$227.025$\\
\hline
0.0062500&$1.812e+00$&$1.720e+00$&$1.312e+00$&$2.718e+00$\\
&$8.155e-01$&$5.997e-01$&$3.927e-01$&$2.718e+00$\\
&$9.982e-01$&$6.534e-01$&$4.057e-01$&$2.722e+00$\\
&$0.253$&$1.229$&$12.632$&$202.192$\\
\hline
0.0031250&$1.819e+00$&$1.742e+00$&$1.351e+00$&$8.268e-01$\\
&$8.162e-01$&$6.031e-01$&$4.005e-01$&$2.257e-01$\\
&$9.995e-01$&$6.575e-01$&$4.139e-01$&$2.279e-01$\\
&$0.488$&$2.180$&$12.696$&$242.456$\\
\hline
0.0015625&$1.823e+00$&$1.752e+00$&$1.371e+00$&$8.494e-01$\\
&$8.165e-01$&$6.048e-01$&$4.045e-01$&$2.298e-01$\\
&$1.000e+00$&$6.595e-01$&$4.182e-01$&$2.321e-01$\\
&$1.027$&$3.962$&$26.062$&$261.074$\\
\hline
\end{tabular}



%
%\include{task2}
%
%\include{task3}
%
%\include{task4}

\end{document}
