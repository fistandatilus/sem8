
\documentclass[12pt,a4paper]{article}
\usepackage[utf8]{inputenc}
\usepackage[OT1]{fontenc}
\usepackage[russian]{babel}
\usepackage{amsmath}
\usepackage{amsfonts}
\usepackage{amssymb}
\usepackage{hyperref}
\hypersetup{linktoc=all}
\usepackage{indentfirst}
\usepackage[left=2cm,right=2cm,top=2cm,bottom=2cm]{geometry}
\usepackage{graphicx}
\usepackage{float}
\usepackage{pdflscape}
\usepackage{diagbox}
\usepackage{}
\author{Борисенков Никита Николаевич}
\title{Отчёт}
\date{}

\newcommand{\xo}{\mathring{x}}
\newcommand{\xx}{x\overline{x}}
\newcommand{\pd}[2]{\dfrac{\partial #1}{\partial #2}}
\DeclareMathOperator*{\mmax}{max}

\begin{document}
\maketitle
\tableofcontents
\newpage
\section{Постановка задачи}

Решается система уравнений, описывающая движение баротропного газа в двумерной области $\Omega$.

Эта система записывается следующим образом:

\begin{gather*}
    \pd{g}{t} + u_1 \pd{g}{x_1} + u_2 \pd{g}{x_2} + \pd{u_1}{x_1} \pd{u_2}{x_2}= 0\\
    \pd{u_k}{t} + u_1 \pd{u_k}{x_1} + u_2 \pd{u_k}{x_2} + p'_{\rho}(\rho) \pd{g}{x_k} =\\
    \dfrac{\mu}{\rho}\left( \dfrac43 \pd{^2 u_k}{x_k^2} + \sum_{m = 1, m \neq k}^2 \left(  \pd{^2 u_k}{x_m^2} + \dfrac13  \pd{^2 u_m}{x_k \partial x_m} \right) \right) + f_k, k = 1,2,\\
    p = p(\rho), g = \ln(\rho),
\end{gather*}

где $\rho$ это плотность газа, а $\textbf{u} = (u_1, u_2)$ вектор его скорости.
Область имеет вид $\Omega = \Omega_{00} \cup \Omega_{10} \cup \Omega_{11} \cup \Omega_{21}$, где $\Omega_{mn} = [m, m+1] \times [n, n+1]$.

\section{Алгоритм}

Для численного приближения решения системы используется последовательная схема с односторонними разностями для ($\ln(\rho), u$)

Уравнения, задающие схему выглядят следующим образом:

\begin{gather*}
    G_t + \delta_1\{\hat{G},V_1\} + \delta_2\{\hat{G},V_2\} + (V_1)_{\xo_1} + (V_2)_{\xo_2} = 0,  \textbf{x} \in \Omega_{\overline{h}}\\
    G_t + V_{kx_k} = 0, \textbf{x} \in \gamma^{-}_k, k = 1, 2; \\
    G_t + V_{k\overline{x}_k} = 0, \textbf{x} \in \gamma^{+}_k, k = 1, 2; \\
    \hat{H}(V_k)_t + \delta_1\{ \hat{V_k}, \hat{H}V_1 \} + \delta_2\{ \hat{V_k}, \hat{H}V_2 \} + p(\hat{H})_{\xo_k} = \\
    \mu\left( \dfrac43 (\hat{V}_k)_{x_k \overline{x}_k} + \sum_{m = 1, m \neq k}^2 (\hat{V}_k)_{x_m \overline{x}_m} + \dfrac13 \sum_{m = 1, m \neq k}^2 (V_m)_{\xo_k \xo_m} \right) + \hat{H}f_k, \textbf{x} \in \Omega_{\overline{h}}\\
    \hat{V_k} = 0, \textbf{x} \in \gamma_h, k = 1, 2
\end{gather*}

По этим уравнениям сначала строится СЛУ на значения $G$ на следующем слое, а с использованием полученных значений $\hat{H}$,
которые выражаются из $\hat{G}$ строится и решается СЛУ на $\hat{V_1}$ и $\hat{V_2}$.

После преобразований, СЛУ схемы имеют следующий вид:
\begin{gather*}
    G_{m_1, m_2}^{n+1}\left( 1 + \dfrac{\tau}{h}(\vert V_{m_1, m_2}^{n} \vert + \vert V_{m_1, m_2}^{n} \vert) \right) +\\
    + G_{m_1 - 1, m_2}^{n + 1} \dfrac{\tau}{2h} (-V_{1 m_1, m_2}^{n} - \vert V_{1 m_1, m_2}^{n}\vert) + G_{m_1 + 1, m_2}^{n + 1} \dfrac{\tau}{2h} (V_{1 m_1, m_2}^{n} - \vert V_{1 m_1, m_2}^{n}\vert) +\\
    + G_{m_1, m_2 - 1}^{n + 1} \dfrac{\tau}{2h} (-V_{2 m_1, m_2}^{n} - \vert V_{2 m_1, m_2}^{n}\vert) + G_{m_1, m_2 + 1}^{n + 1} \dfrac{\tau}{2h} (V_{2 m_1, m_2}^{n} - \vert V_{2 m_1, m_2}^{n}\vert) =\\
    G_{m_1, m_2}^{n} - \dfrac{\tau}{2h} ( V_{1 m_1 + 1, m_2}^{n} - V_{1 m_1 - 1, m_2}^{n} + V_{2 m_1, m_2 + 1}^{n} - V_{2 m_1, m_2 - 1}^{n} ), (m_1h, m_2h) \text{ -- внутренняя точка}  \\ 
    G_{m_1, m_2}^{n+1} = G_{m_1, m_2}^{n} - \dfrac{\tau}{h} ( V_{1 m_1 + 1, m_2}^{n} - V_{1 m_1, m_2}^{n}, (m_1h, m_2h) \text{ -- точка левой границы}  \\ 
    G_{m_1, m_2}^{n+1} = G_{m_1, m_2}^{n} - \dfrac{\tau}{h} ( V_{1 m_1, m_2}^{n} - V_{1 m_1 - 1, m_2}^{n}, (m_1h, m_2h) \text{ -- точка правой границы}  \\ 
    G_{m_1, m_2}^{n+1} = G_{m_1, m_2}^{n} - \dfrac{\tau}{h} ( V_{1 m_1, m_2 + 1}^{n} - V_{1 m_1, m_2}^{n}, (m_1h, m_2h) \text{ -- точка верхней границы}  \\ 
    G_{m_1, m_2}^{n+1} = G_{m_1, m_2}^{n} - \dfrac{\tau}{h} ( V_{1 m_1, m_2}^{n} - V_{1 m_1, m_2 - 1}^{n}, (m_1h, m_2h) \text{ -- точка нижней границы}  \\ 
    V_{1 m_1, m_2}^{n+1} \left(H_{m_1, m_2}^{n+1} + \dfrac{\tau}{h} \left( \vert H_{m_1, m_2}^{n+1} V_{1 m_1, m_2}^{n} \vert  + \vert H_{m_1, m_2}^{n+1} V_{2 m_1, m_2}^{n} \vert + \dfrac{14}{3} \dfrac{\mu}{h} \right) \right) +\\
    + V_{1 m_1 - 1, m_2}^{n + 1} \dfrac{\tau}{2h} \left(-H_{m_1, m_2}^{n+1}V_{1 m_1, m_2}^{n} - \vert H_{m_1, m_2}^{n+1}V_{1 m_1, m_2}^{n}\vert - \dfrac43 \dfrac{\mu}{h}\right) +\\
    + V_{1 m_1 + 1, m_2}^{n + 1} \dfrac{\tau}{2h} \left(H_{m_1, m_2}^{n+1}V_{1 m_1, m_2}^{n} - \vert H_{m_1, m_2}^{n+1}V_{1 m_1, m_2}^{n}\vert - \dfrac43 \dfrac{\mu}{h}\right) +\\
    + V_{1 m_1, m_2 - 1}^{n + 1} \dfrac{\tau}{2h} \left(-H_{m_1, m_2}^{n+1}V_{2 m_1, m_2}^{n} - \vert H_{m_1, m_2}^{n+1}V_{2 m_1, m_2}^{n}\vert - \dfrac{\mu}{h}\right) +\\
    + V_{1 m_1, m_2 + 1}^{n + 1} \dfrac{\tau}{2h} \left(H_{m_1, m_2}^{n+1}V_{2 m_1, m_2}^{n} - \vert H_{m_1, m_2}^{n+1}V_{2 m_1, m_2}^{n}\vert - \dfrac{\mu}{h}\right) =\\
    = H_{m_1, m_2}^{n+1}V_{1 m_1, m_2}^{n} - \tau \dfrac{1}{2h}(P(H_{m_1 + 1, m_2}^{n+1}) - P(H_{m_1 - 1, m_2}^{n+1})) +\\
    +\dfrac{1}{12} \dfrac{\tau\mu}{h^2} ( V_{2 m_1 + 1, m_2 + 1}^{n} - V_{2 m_1 + 1, m_2 + 1}^{n} - V_{2 m_1 - 1, m_2 + 1}^{n} + V_{2 m_1 - 1, m_2 - 1}^{n} ) + \tau H_{m_1, m_2}^{n+1}f_{1 m_1, m_2}^{n}\\
    V_{2 m_1, m_2}^{n+1} \left(H_{m_1, m_2}^{n+1} + \dfrac{\tau}{h} \left( \vert H_{m_1, m_2}^{n+1} V_{1 m_1, m_2}^{n} \vert  + \vert H_{m_1, m_2}^{n+1} V_{2 m_1, m_2}^{n} \vert + \dfrac{14}{3} \dfrac{\mu}{h} \right) \right) +\\
    + V_{1 m_1 - 1, m_2}^{n + 1} \dfrac{\tau}{2h} \left(-H_{m_1, m_2}^{n+1}V_{1 m_1, m_2}^{n} - \vert H_{m_1, m_2}^{n+1}V_{1 m_1, m_2}^{n}\vert - \dfrac{\mu}{h}\right) +\\
    + V_{1 m_1 + 1, m_2}^{n + 1} \dfrac{\tau}{2h} \left(H_{m_1, m_2}^{n+1}V_{1 m_1, m_2}^{n} - \vert H_{m_1, m_2}^{n+1}V_{1 m_1, m_2}^{n}\vert - \dfrac{\mu}{h}\right) +\\
    + V_{1 m_1, m_2 - 1}^{n + 1} \dfrac{\tau}{2h} \left(-H_{m_1, m_2}^{n+1}V_{2 m_1, m_2}^{n} - \vert H_{m_1, m_2}^{n+1}V_{2 m_1, m_2}^{n}\vert - \dfrac43 \dfrac{\mu}{h}\right) +\\
    + V_{1 m_1, m_2 + 1}^{n + 1} \dfrac{\tau}{2h} \left(H_{m_1, m_2}^{n+1}V_{2 m_1, m_2}^{n} - \vert H_{m_1, m_2}^{n+1}V_{2 m_1, m_2}^{n}\vert - \dfrac43 \dfrac{\mu}{h}\right) =\\
    = H_{m_1, m_2}^{n+1}V_{2 m_1, m_2}^{n} - \tau \dfrac{1}{2h}(P(H_{m_1 + 1, m_2}^{n+1}) - P(H_{m_1 - 1, m_2}^{n+1})) +\\
    +\dfrac{1}{12} \dfrac{\tau\mu}{h^2} ( V_{1 m_1 + 1, m_2 + 1}^{n} - V_{1 m_1 + 1, m_2 + 1}^{n} - V_{1 m_1 - 1, m_2 + 1}^{n} + V_{1 m_1 - 1, m_2 - 1}^{n} ) + \tau H_{m_1, m_2}^{n+1}f_{1 m_1, m_2}^{n}\\
    V_{km_1, m_2}^{n+1} = 0, k = 1,2 \text{ для граничных точек}
\end{gather*}

\section{Отладочный тест}

Для проверки правильности написанной схемы, она применялась с начальными условиями и правыми частями, соответствующими следующим гладким решениям:
\begin{equation*}
    \rho(t, x) = e^{t + x_1 - x_2}, u_1(t, x) = \sin(2\pi x_1)\sin(2\pi x_2)e^{-t}, u_2 = \sin(2\pi x_1)\sin(2\pi x_2)e^{t}
\end{equation*}

\subsection{Результаты на разных параметрах}

В следующих таблицах приведены результаты работы программы при разных значениях параметров схемы и задачи.

\subsubsection{$\mu = 0.100, p(\rho) = \rho$}
$$\Vert g - \ln(\rho)\Vert$$
\begin{tabular}{*{5}{|c}|}
\hline
\diagbox{$\tau$}{$h$}&0.050000&0.025000&0.012500&0.0062500\\
\hline
0.050000&$1.088e+00$&$4.445e-01$&$1.976e-01$&$5.384e+00$\\
&$4.949e-01$&$2.554e-01$&$1.511e-01$&$1.396e-01$\\
&$5.563e-01$&$2.628e-01$&$1.519e-01$&$1.554e-01$\\
&$0.070$&$0.730$&$12.232$&$279.793$\\
\hline
0.025000&$1.258e+00$&$5.310e-01$&$2.511e-01$&$1.162e-01$\\
&$4.917e-01$&$2.435e-01$&$1.322e-01$&$7.851e-02$\\
&$5.660e-01$&$2.535e-01$&$1.334e-01$&$7.863e-02$\\
&$0.099$&$0.736$&$12.748$&$306.916$\\
\hline
0.012500&$1.350e+00$&$5.788e-01$&$2.820e-01$&$1.370e-01$\\
&$4.920e-01$&$2.397e-01$&$1.252e-01$&$6.808e-02$\\
&$5.738e-01$&$2.515e-01$&$1.269e-01$&$6.828e-02$\\
&$0.163$&$1.189$&$18.116$&$334.376$\\
\hline
0.0062500&$1.398e+00$&$6.041e-01$&$2.985e-01$&$1.488e-01$\\
&$4.928e-01$&$2.385e-01$&$1.228e-01$&$6.420e-02$\\
&$5.786e-01$&$2.513e-01$&$1.247e-01$&$6.446e-02$\\
&$0.276$&$1.785$&$26.031$&$260.138$\\
\hline
\end{tabular}
$$\Vert v_1 - u_1 \Vert$$
\begin{tabular}{*{5}{|c}|}
\hline
\diagbox{$\tau$}{$h$}&0.050000&0.025000&0.012500&0.0062500\\
\hline
0.050000&$3.940e-01$&$2.222e-01$&$1.425e-01$&$5.857e-01$\\
&$2.220e-01$&$1.274e-01$&$7.943e-02$&$6.542e-02$\\
&$2.538e-01$&$1.329e-01$&$8.031e-02$&$6.719e-02$\\
&$0.070$&$0.730$&$12.232$&$279.793$\\
\hline
0.025000&$4.209e-01$&$2.309e-01$&$1.210e-01$&$7.169e-02$\\
&$2.166e-01$&$1.188e-01$&$6.644e-02$&$4.063e-02$\\
&$2.494e-01$&$1.248e-01$&$6.737e-02$&$4.076e-02$\\
&$0.099$&$0.736$&$12.748$&$306.916$\\
\hline
0.012500&$4.311e-01$&$2.363e-01$&$1.241e-01$&$6.378e-02$\\
&$2.147e-01$&$1.159e-01$&$6.198e-02$&$3.399e-02$\\
&$2.483e-01$&$1.223e-01$&$6.301e-02$&$3.413e-02$\\
&$0.163$&$1.189$&$18.116$&$334.376$\\
\hline
0.0062500&$4.356e-01$&$2.396e-01$&$1.263e-01$&$6.519e-02$\\
&$2.140e-01$&$1.148e-01$&$6.044e-02$&$3.172e-02$\\
&$2.481e-01$&$1.215e-01$&$6.153e-02$&$3.187e-02$\\
&$0.276$&$1.785$&$26.031$&$260.138$\\
\hline
\end{tabular}
$$\Vert v_2 - u_2\Vert$$
\begin{tabular}{*{5}{|c}|}
\hline
\diagbox{$\tau$}{$h$}&0.050000&0.025000&0.012500&0.0062500\\
\hline
0.050000&$8.106e-01$&$4.844e-01$&$2.592e-01$&$6.827e-01$\\
&$4.315e-01$&$2.550e-01$&$1.418e-01$&$9.679e-02$\\
&$4.932e-01$&$2.650e-01$&$1.431e-01$&$9.843e-02$\\
&$0.070$&$0.730$&$12.232$&$279.793$\\
\hline
0.025000&$9.106e-01$&$5.588e-01$&$3.008e-01$&$1.503e-01$\\
&$4.498e-01$&$2.672e-01$&$1.445e-01$&$7.566e-02$\\
&$5.168e-01$&$2.784e-01$&$1.460e-01$&$7.585e-02$\\
&$0.099$&$0.736$&$12.748$&$306.916$\\
\hline
0.012500&$9.650e-01$&$6.005e-01$&$3.363e-01$&$1.741e-01$\\
&$4.606e-01$&$2.759e-01$&$1.503e-01$&$7.713e-02$\\
&$5.307e-01$&$2.878e-01$&$1.521e-01$&$7.736e-02$\\
&$0.163$&$1.189$&$18.116$&$334.376$\\
\hline
0.0062500&$9.934e-01$&$6.226e-01$&$3.557e-01$&$1.884e-01$\\
&$4.664e-01$&$2.808e-01$&$1.544e-01$&$7.994e-02$\\
&$5.381e-01$&$2.933e-01$&$1.562e-01$&$8.019e-02$\\
&$0.276$&$1.785$&$26.031$&$260.138$\\
\hline
\end{tabular}


\subsubsection{$\mu = 0.010, p(\rho) = \rho$}
$$\Vert g - \ln(\rho)\Vert$$
\begin{tabular}{*{5}{|c}|}
\hline
\diagbox{$\tau$}{$h$}&0.050000&0.025000&0.012500&0.0062500\\
\hline
0.050000&$1.792e+00$&$7.973e-01$&$5.209e+00$&$2.638e+11$\\
&$6.916e-01$&$3.546e-01$&$2.152e+00$&$5.481e+10$\\
&$8.232e-01$&$3.785e-01$&$2.224e+00$&$6.382e+10$\\
&$0.060$&$0.492$&$5.532$&$152.944$\\
\hline
0.025000&$1.948e+00$&$9.868e-01$&$4.912e-01$&$1.063e+43$\\
&$7.106e-01$&$3.686e-01$&$1.982e-01$&$1.410e+41$\\
&$8.670e-01$&$4.009e-01$&$2.061e-01$&$2.648e+41$\\
&$0.083$&$0.528$&$5.445$&$187.363$\\
\hline
0.012500&$2.031e+00$&$1.081e+00$&$5.825e-01$&$4.186e-01$\\
&$7.212e-01$&$3.822e-01$&$2.043e-01$&$1.075e-01$\\
&$8.885e-01$&$4.193e-01$&$2.111e-01$&$1.109e-01$\\
&$0.159$&$0.803$&$7.380$&$202.153$\\
\hline
0.0062500&$2.074e+00$&$1.129e+00$&$6.345e-01$&$3.063e-01$\\
&$7.270e-01$&$3.906e-01$&$2.103e-01$&$1.079e-01$\\
&$9.002e-01$&$4.301e-01$&$2.178e-01$&$1.091e-01$\\
&$0.257$&$1.224$&$10.561$&$208.245$\\
\hline
\end{tabular}
$$\Vert v_1 - u_1 \Vert$$
\begin{tabular}{*{5}{|c}|}
\hline
\diagbox{$\tau$}{$h$}&0.050000&0.025000&0.012500&0.0062500\\
\hline
0.050000&$8.893e-01$&$6.627e-01$&$1.547e+01$&$3.679e-01$\\
&$4.108e-01$&$2.546e-01$&$1.209e+00$&$3.679e-01$\\
&$5.161e-01$&$2.860e-01$&$1.519e+00$&$3.684e-01$\\
&$0.060$&$0.492$&$5.532$&$152.944$\\
\hline
0.025000&$9.139e-01$&$7.130e-01$&$4.622e-01$&$3.679e-01$\\
&$4.207e-01$&$2.631e-01$&$1.552e-01$&$3.679e-01$\\
&$5.354e-01$&$2.983e-01$&$1.660e-01$&$3.684e-01$\\
&$0.083$&$0.528$&$5.445$&$187.363$\\
\hline
0.012500&$9.392e-01$&$7.434e-01$&$4.980e-01$&$3.063e-01$\\
&$4.265e-01$&$2.699e-01$&$1.599e-01$&$8.772e-02$\\
&$5.443e-01$&$3.074e-01$&$1.696e-01$&$9.077e-02$\\
&$0.159$&$0.803$&$7.380$&$202.153$\\
\hline
0.0062500&$9.739e-01$&$7.584e-01$&$5.167e-01$&$3.293e-01$\\
&$4.300e-01$&$2.739e-01$&$1.636e-01$&$8.996e-02$\\
&$5.490e-01$&$3.126e-01$&$1.737e-01$&$9.191e-02$\\
&$0.257$&$1.224$&$10.561$&$208.245$\\
\hline
\end{tabular}
$$\Vert v_2 - u_2\Vert$$
\begin{tabular}{*{5}{|c}|}
\hline
\diagbox{$\tau$}{$h$}&0.050000&0.025000&0.012500&0.0062500\\
\hline
0.050000&$1.810e+00$&$1.245e+00$&$2.703e+00$&$2.718e+00$\\
&$8.762e-01$&$5.343e-01$&$1.183e+00$&$2.718e+00$\\
&$1.048e+00$&$5.675e-01$&$1.215e+00$&$2.722e+00$\\
&$0.060$&$0.492$&$5.532$&$152.944$\\
\hline
0.025000&$2.233e+00$&$1.443e+00$&$8.806e-01$&$2.718e+00$\\
&$9.181e-01$&$5.758e-01$&$3.205e-01$&$2.718e+00$\\
&$1.107e+00$&$6.143e-01$&$3.276e-01$&$2.722e+00$\\
&$0.083$&$0.528$&$5.445$&$187.363$\\
\hline
0.012500&$2.513e+00$&$1.539e+00$&$1.001e+00$&$5.105e-01$\\
&$9.453e-01$&$5.995e-01$&$3.419e-01$&$1.771e-01$\\
&$1.149e+00$&$6.408e-01$&$3.493e-01$&$1.788e-01$\\
&$0.159$&$0.803$&$7.380$&$202.153$\\
\hline
0.0062500&$2.574e+00$&$1.581e+00$&$1.065e+00$&$5.646e-01$\\
&$9.583e-01$&$6.119e-01$&$3.545e-01$&$1.864e-01$\\
&$1.167e+00$&$6.546e-01$&$3.624e-01$&$1.876e-01$\\
&$0.257$&$1.224$&$10.561$&$208.245$\\
\hline
\end{tabular}


\subsubsection{$\mu = 0.001, p(\rho) = \rho$}
$$\Vert g - \ln(\rho)\Vert$$
\begin{tabular}{*{5}{|c}|}
\hline
\diagbox{$\tau$}{$h$}&0.050000&0.025000&0.012500&0.0062500\\
\hline
0.050000&$2.117e+00$&$4.554e+00$&$7.321e+00$&$7.291e+11$\\
&$7.585e-01$&$3.458e+00$&$3.622e+00$&$3.197e+10$\\
&$9.206e-01$&$3.480e+00$&$3.851e+00$&$4.355e+10$\\
&$0.048$&$0.423$&$5.192$&$116.441$\\
\hline
0.025000&$2.307e+00$&$1.263e+00$&$7.706e+17$&$4.937e+47$\\
&$7.796e-01$&$4.310e-01$&$8.655e+16$&$3.628e+45$\\
&$9.655e-01$&$4.772e-01$&$1.556e+17$&$6.284e+45$\\
&$0.075$&$0.542$&$8.780$&$153.487$\\
\hline
0.012500&$2.410e+00$&$1.387e+00$&$8.963e-01$&$1.191e+76$\\
&$7.932e-01$&$4.536e-01$&$2.500e-01$&$1.295e+74$\\
&$9.895e-01$&$5.050e-01$&$2.607e-01$&$2.480e+74$\\
&$0.157$&$0.824$&$8.381$&$141.133$\\
\hline
0.0062500&$2.465e+00$&$1.454e+00$&$9.573e-01$&$4.284e-01$\\
&$8.018e-01$&$4.661e-01$&$2.616e-01$&$1.321e-01$\\
&$1.004e+00$&$5.200e-01$&$2.734e-01$&$1.340e-01$\\
&$0.255$&$1.185$&$9.230$&$161.027$\\
\hline
\end{tabular}
$$\Vert v_1 - u_1 \Vert$$
\begin{tabular}{*{5}{|c}|}
\hline
\diagbox{$\tau$}{$h$}&0.050000&0.025000&0.012500&0.0062500\\
\hline
0.050000&$9.817e-01$&$9.039e-01$&$4.461e+04$&$3.679e-01$\\
&$4.709e-01$&$7.868e-01$&$2.379e+03$&$3.679e-01$\\
&$6.081e-01$&$8.304e-01$&$3.308e+03$&$3.684e-01$\\
&$0.048$&$0.423$&$5.192$&$116.441$\\
\hline
0.025000&$1.036e+00$&$8.759e-01$&$3.679e-01$&$3.679e-01$\\
&$4.845e-01$&$3.223e-01$&$3.679e-01$&$3.679e-01$\\
&$6.276e-01$&$3.766e-01$&$3.701e-01$&$3.684e-01$\\
&$0.075$&$0.542$&$8.780$&$153.487$\\
\hline
0.012500&$1.114e+00$&$9.176e-01$&$6.561e-01$&$3.679e-01$\\
&$4.923e-01$&$3.319e-01$&$2.120e-01$&$3.679e-01$\\
&$6.387e-01$&$3.894e-01$&$2.314e-01$&$3.684e-01$\\
&$0.157$&$0.824$&$8.381$&$141.133$\\
\hline
0.0062500&$1.138e+00$&$9.453e-01$&$6.677e-01$&$5.116e-01$\\
&$4.963e-01$&$3.373e-01$&$2.171e-01$&$1.293e-01$\\
&$6.441e-01$&$3.965e-01$&$2.372e-01$&$1.348e-01$\\
&$0.255$&$1.185$&$9.230$&$161.027$\\
\hline
\end{tabular}
$$\Vert v_2 - u_2\Vert$$
\begin{tabular}{*{5}{|c}|}
\hline
\diagbox{$\tau$}{$h$}&0.050000&0.025000&0.012500&0.0062500\\
\hline
0.050000&$2.338e+00$&$3.202e+00$&$2.389e+02$&$2.718e+00$\\
&$9.951e-01$&$2.200e+00$&$4.410e+00$&$2.718e+00$\\
&$1.216e+00$&$2.281e+00$&$6.914e+00$&$2.722e+00$\\
&$0.048$&$0.423$&$5.192$&$116.441$\\
\hline
0.025000&$2.633e+00$&$1.774e+00$&$2.718e+00$&$2.718e+00$\\
&$1.042e+00$&$6.793e-01$&$2.718e+00$&$2.718e+00$\\
&$1.285e+00$&$7.351e-01$&$2.735e+00$&$2.722e+00$\\
&$0.075$&$0.542$&$8.780$&$153.487$\\
\hline
0.012500&$2.683e+00$&$1.862e+00$&$1.338e+00$&$2.718e+00$\\
&$1.072e+00$&$7.055e-01$&$4.231e-01$&$2.718e+00$\\
&$1.326e+00$&$7.656e-01$&$4.361e-01$&$2.722e+00$\\
&$0.157$&$0.824$&$8.381$&$141.133$\\
\hline
0.0062500&$2.726e+00$&$1.898e+00$&$1.412e+00$&$8.025e-01$\\
&$1.095e+00$&$7.189e-01$&$4.391e-01$&$2.377e-01$\\
&$1.360e+00$&$7.816e-01$&$4.530e-01$&$2.399e-01$\\
&$0.255$&$1.185$&$9.230$&$161.027$\\
\hline
\end{tabular}


\subsubsection{$\mu = 0.100, p(\rho) = 10\rho$}
$$\Vert g - \ln(\rho)\Vert$$
\begin{tabular}{*{5}{|c}|}
\hline
\diagbox{$\tau$}{$h$}&0.050000&0.025000&0.012500&0.0062500\\
\hline
0.050000&$8.428e+01$&$3.519e+05$&$2.942e+08$&$2.130e+04$\\
&$6.561e+00$&$3.893e+04$&$1.818e+08$&$6.425e+02$\\
&$8.161e+00$&$4.952e+04$&$1.845e+08$&$7.244e+02$\\
&$0.101$&$1.060$&$8.553$&$227.868$\\
\hline
0.025000&$1.272e-01$&$1.917e+03$&$3.074e+10$&$1.397e+03$\\
&$1.510e-01$&$3.880e+01$&$6.449e+09$&$9.859e+00$\\
&$1.554e-01$&$8.132e+01$&$8.583e+09$&$1.609e+01$\\
&$0.094$&$0.782$&$5.436$&$282.897$\\
\hline
0.012500&$1.268e-01$&$7.121e-02$&$2.987e+00$&$2.813e+00$\\
&$1.509e-01$&$7.734e-02$&$2.449e+00$&$2.527e+00$\\
&$1.555e-01$&$7.802e-02$&$2.450e+00$&$2.527e+00$\\
&$0.180$&$1.020$&$19.376$&$278.382$\\
\hline
0.0062500&$1.266e-01$&$7.072e-02$&$3.796e-02$&$4.250e+01$\\
&$1.508e-01$&$7.731e-02$&$4.031e-02$&$3.363e-01$\\
&$1.556e-01$&$7.802e-02$&$4.040e-02$&$4.483e-01$\\
&$0.289$&$1.559$&$17.764$&$307.451$\\
\hline
\end{tabular}
$$\Vert v_1 - u_1 \Vert$$
\begin{tabular}{*{5}{|c}|}
\hline
\diagbox{$\tau$}{$h$}&0.050000&0.025000&0.012500&0.0062500\\
\hline
0.050000&$1.384e+01$&$1.112e+01$&$3.679e-01$&$3.802e+00$\\
&$4.858e+00$&$3.859e+00$&$3.679e-01$&$3.620e+00$\\
&$5.362e+00$&$3.991e+00$&$3.701e-01$&$3.622e+00$\\
&$0.101$&$1.060$&$8.553$&$227.868$\\
\hline
0.025000&$6.021e-01$&$5.666e+00$&$3.679e-01$&$8.355e+01$\\
&$3.376e-01$&$3.716e+00$&$3.679e-01$&$3.265e+00$\\
&$3.906e-01$&$3.794e+00$&$3.701e-01$&$3.388e+00$\\
&$0.094$&$0.782$&$5.436$&$282.897$\\
\hline
0.012500&$6.052e-01$&$3.639e-01$&$3.154e+00$&$2.617e+00$\\
&$3.356e-01$&$1.883e-01$&$2.198e+00$&$2.475e+00$\\
&$3.896e-01$&$1.982e-01$&$2.205e+00$&$2.476e+00$\\
&$0.180$&$1.020$&$19.376$&$278.382$\\
\hline
0.0062500&$6.068e-01$&$3.646e-01$&$1.991e-01$&$2.526e+00$\\
&$3.348e-01$&$1.871e-01$&$1.002e-01$&$6.350e-01$\\
&$3.892e-01$&$1.971e-01$&$1.018e-01$&$6.480e-01$\\
&$0.289$&$1.559$&$17.764$&$307.451$\\
\hline
\end{tabular}
$$\Vert v_2 - u_2\Vert$$
\begin{tabular}{*{5}{|c}|}
\hline
\diagbox{$\tau$}{$h$}&0.050000&0.025000&0.012500&0.0062500\\
\hline
0.050000&$4.784e+00$&$3.944e+03$&$2.718e+00$&$4.742e+00$\\
&$1.994e+00$&$4.803e+02$&$2.718e+00$&$3.215e+00$\\
&$2.369e+00$&$5.787e+02$&$2.735e+00$&$3.218e+00$\\
&$0.101$&$1.060$&$8.553$&$227.868$\\
\hline
0.025000&$6.661e-01$&$9.620e+05$&$2.718e+00$&$4.130e+00$\\
&$3.545e-01$&$3.025e+04$&$2.718e+00$&$1.491e+00$\\
&$4.071e-01$&$5.149e+04$&$2.735e+00$&$1.498e+00$\\
&$0.094$&$0.782$&$5.436$&$282.897$\\
\hline
0.012500&$6.852e-01$&$4.096e-01$&$1.946e+00$&$2.904e+00$\\
&$3.577e-01$&$2.165e-01$&$1.129e+00$&$2.284e+00$\\
&$4.120e-01$&$2.255e-01$&$1.135e+00$&$2.285e+00$\\
&$0.180$&$1.020$&$19.376$&$278.382$\\
\hline
0.0062500&$6.947e-01$&$4.183e-01$&$2.312e-01$&$3.991e+00$\\
&$3.594e-01$&$2.179e-01$&$1.199e-01$&$5.621e-01$\\
&$4.146e-01$&$2.271e-01$&$1.212e-01$&$5.791e-01$\\
&$0.289$&$1.559$&$17.764$&$307.451$\\
\hline
\end{tabular}


\subsubsection{$\mu = 0.010, p(\rho) = 10\rho$}
$$\Vert g - \ln(\rho)\Vert$$
\begin{tabular}{*{5}{|c}|}
\hline
\diagbox{$\tau$}{$h$}&0.050000&0.025000&0.012500&0.0062500\\
\hline
0.050000&$1.897e+47$&$3.599e+00$&$3.527e+00$&$3.528e+00$\\
&$4.005e+46$&$3.364e+00$&$3.281e+00$&$3.281e+00$\\
&$7.094e+46$&$3.376e+00$&$3.282e+00$&$3.281e+00$\\
&$0.128$&$0.709$&$9.877$&$181.904$\\
\hline
0.025000&$1.459e-01$&$2.094e+23$&$1.257e+19$&$3.171e+21$\\
&$1.519e-01$&$8.066e+21$&$1.558e+18$&$5.146e+20$\\
&$1.611e-01$&$1.437e+22$&$2.106e+18$&$7.144e+20$\\
&$0.078$&$0.841$&$5.112$&$176.183$\\
\hline
0.012500&$1.556e-01$&$1.364e-01$&$8.221e+18$&$1.961e+10$\\
&$1.529e-01$&$7.938e-02$&$1.856e+18$&$1.496e+09$\\
&$1.626e-01$&$8.133e-02$&$2.197e+18$&$1.733e+09$\\
&$0.163$&$0.860$&$10.043$&$157.727$\\
\hline
0.0062500&$1.601e-01$&$1.476e-01$&$7.909e-02$&$1.530e+56$\\
&$1.534e-01$&$7.997e-02$&$4.163e-02$&$2.494e+54$\\
&$1.634e-01$&$8.203e-02$&$4.196e-02$&$3.906e+54$\\
&$0.257$&$1.227$&$10.858$&$146.444$\\
\hline
\end{tabular}
$$\Vert v_1 - u_1 \Vert$$
\begin{tabular}{*{5}{|c}|}
\hline
\diagbox{$\tau$}{$h$}&0.050000&0.025000&0.012500&0.0062500\\
\hline
0.050000&$3.679e-01$&$4.721e+00$&$2.968e+00$&$2.970e+00$\\
&$3.679e-01$&$2.870e+00$&$2.450e+00$&$2.453e+00$\\
&$4.023e-01$&$3.062e+00$&$2.484e+00$&$2.462e+00$\\
&$0.128$&$0.709$&$9.877$&$181.904$\\
\hline
0.025000&$1.134e+00$&$3.679e-01$&$3.679e-01$&$3.679e-01$\\
&$5.406e-01$&$3.679e-01$&$3.679e-01$&$3.679e-01$\\
&$6.785e-01$&$3.768e-01$&$3.701e-01$&$3.684e-01$\\
&$0.078$&$0.841$&$5.112$&$176.183$\\
\hline
0.012500&$1.152e+00$&$9.201e-01$&$3.679e-01$&$3.679e-01$\\
&$5.389e-01$&$3.398e-01$&$3.679e-01$&$3.679e-01$\\
&$6.811e-01$&$3.812e-01$&$3.701e-01$&$3.684e-01$\\
&$0.163$&$0.860$&$10.043$&$157.727$\\
\hline
0.0062500&$1.160e+00$&$9.442e-01$&$7.073e-01$&$3.679e-01$\\
&$5.381e-01$&$3.410e-01$&$2.004e-01$&$3.679e-01$\\
&$6.825e-01$&$3.838e-01$&$2.104e-01$&$3.684e-01$\\
&$0.257$&$1.227$&$10.858$&$146.444$\\
\hline
\end{tabular}
$$\Vert v_2 - u_2\Vert$$
\begin{tabular}{*{5}{|c}|}
\hline
\diagbox{$\tau$}{$h$}&0.050000&0.025000&0.012500&0.0062500\\
\hline
0.050000&$2.718e+00$&$3.094e+00$&$1.750e+00$&$1.753e+00$\\
&$2.718e+00$&$2.398e+00$&$1.350e+00$&$1.351e+00$\\
&$2.972e+00$&$2.571e+00$&$1.363e+00$&$1.354e+00$\\
&$0.128$&$0.709$&$9.877$&$181.904$\\
\hline
0.025000&$1.266e+00$&$2.718e+00$&$2.718e+00$&$2.718e+00$\\
&$6.026e-01$&$2.718e+00$&$2.718e+00$&$2.718e+00$\\
&$7.416e-01$&$2.784e+00$&$2.735e+00$&$2.722e+00$\\
&$0.078$&$0.841$&$5.112$&$176.183$\\
\hline
0.012500&$1.291e+00$&$1.269e+00$&$2.718e+00$&$2.718e+00$\\
&$6.063e-01$&$4.180e-01$&$2.718e+00$&$2.718e+00$\\
&$7.495e-01$&$4.547e-01$&$2.735e+00$&$2.722e+00$\\
&$0.163$&$0.860$&$10.043$&$157.727$\\
\hline
0.0062500&$1.301e+00$&$1.297e+00$&$9.542e-01$&$2.718e+00$\\
&$6.075e-01$&$4.233e-01$&$2.612e-01$&$2.718e+00$\\
&$7.526e-01$&$4.613e-01$&$2.686e-01$&$2.722e+00$\\
&$0.257$&$1.227$&$10.858$&$146.444$\\
\hline
\end{tabular}


\subsubsection{$\mu = 0.001, p(\rho) = 10\rho$}
$$\Vert g - \ln(\rho)\Vert$$
\begin{tabular}{*{5}{|c}|}
\hline
\diagbox{$\tau$}{$h$}&0.050000&0.025000&0.012500&0.0062500\\
\hline
0.050000&$2.208e+25$&$6.004e+09$&$3.178e+04$&$3.000e+00$\\
&$5.048e+24$&$4.306e+08$&$3.121e+03$&$3.266e+00$\\
&$7.992e+24$&$5.701e+08$&$3.547e+03$&$3.266e+00$\\
&$0.187$&$1.422$&$23.396$&$186.238$\\
\hline
0.025000&$1.806e-01$&$3.646e+00$&$8.741e+72$&$4.981e+34$\\
&$1.569e-01$&$3.069e+00$&$1.805e+72$&$2.759e+33$\\
&$1.684e-01$&$3.082e+00$&$2.420e+72$&$4.165e+33$\\
&$0.081$&$1.145$&$8.690$&$133.268$\\
\hline
0.012500&$1.896e-01$&$6.836e-01$&$6.064e+02$&$3.000e+00$\\
&$1.579e-01$&$3.502e-01$&$3.093e+01$&$3.266e+00$\\
&$1.698e-01$&$3.902e-01$&$4.977e+01$&$3.266e+00$\\
&$0.160$&$0.822$&$15.625$&$157.465$\\
\hline
0.0062500&$1.933e-01$&$2.094e-01$&$2.031e+00$&$2.552e+45$\\
&$1.584e-01$&$8.578e-02$&$1.149e+00$&$2.091e+44$\\
&$1.705e-01$&$8.905e-02$&$1.162e+00$&$2.158e+44$\\
&$0.258$&$1.244$&$12.289$&$183.693$\\
\hline
\end{tabular}
$$\Vert v_1 - u_1 \Vert$$
\begin{tabular}{*{5}{|c}|}
\hline
\diagbox{$\tau$}{$h$}&0.050000&0.025000&0.012500&0.0062500\\
\hline
0.050000&$0.000e+00$&$3.679e-01$&$1.038e+15$&$3.679e-01$\\
&$-nan$&$3.679e-01$&$2.203e+13$&$3.679e-01$\\
&$-nan$&$3.768e-01$&$3.871e+13$&$3.684e-01$\\
&$0.187$&$1.422$&$23.396$&$186.238$\\
\hline
0.025000&$1.275e+00$&$3.662e+00$&$3.679e-01$&$3.679e-01$\\
&$5.865e-01$&$2.446e+00$&$3.679e-01$&$3.679e-01$\\
&$7.567e-01$&$2.721e+00$&$3.701e-01$&$3.684e-01$\\
&$0.081$&$1.145$&$8.690$&$133.268$\\
\hline
0.012500&$1.300e+00$&$1.888e+00$&$0.000e+00$&$3.679e-01$\\
&$5.835e-01$&$1.109e+00$&$-nan$&$3.679e-01$\\
&$7.581e-01$&$1.261e+00$&$-nan$&$3.684e-01$\\
&$0.160$&$0.822$&$15.625$&$157.465$\\
\hline
0.0062500&$1.302e+00$&$1.123e+00$&$4.247e+00$&$3.679e-01$\\
&$5.818e-01$&$3.913e-01$&$2.241e+00$&$3.679e-01$\\
&$7.587e-01$&$4.552e-01$&$2.319e+00$&$3.684e-01$\\
&$0.258$&$1.244$&$12.289$&$183.693$\\
\hline
\end{tabular}
$$\Vert v_2 - u_2\Vert$$
\begin{tabular}{*{5}{|c}|}
\hline
\diagbox{$\tau$}{$h$}&0.050000&0.025000&0.012500&0.0062500\\
\hline
0.050000&$0.000e+00$&$2.718e+00$&$9.845e+14$&$2.718e+00$\\
&$-nan$&$2.718e+00$&$2.608e+13$&$2.718e+00$\\
&$-nan$&$2.784e+00$&$4.258e+13$&$2.722e+00$\\
&$0.187$&$1.422$&$23.396$&$186.238$\\
\hline
0.025000&$1.379e+00$&$4.017e+00$&$2.718e+00$&$2.718e+00$\\
&$6.573e-01$&$1.710e+00$&$2.718e+00$&$2.718e+00$\\
&$8.265e-01$&$1.949e+00$&$2.735e+00$&$2.722e+00$\\
&$0.081$&$1.145$&$8.690$&$133.268$\\
\hline
0.012500&$1.389e+00$&$2.737e+00$&$0.000e+00$&$2.718e+00$\\
&$6.574e-01$&$1.156e+00$&$-nan$&$2.718e+00$\\
&$8.300e-01$&$1.244e+00$&$-nan$&$2.722e+00$\\
&$0.160$&$0.822$&$15.625$&$157.465$\\
\hline
0.0062500&$1.391e+00$&$1.486e+00$&$5.093e+00$&$2.718e+00$\\
&$6.563e-01$&$4.888e-01$&$2.453e+00$&$2.718e+00$\\
&$8.303e-01$&$5.446e-01$&$2.507e+00$&$2.722e+00$\\
&$0.258$&$1.244$&$12.289$&$183.693$\\
\hline
\end{tabular}


\subsubsection{$\mu = 0.100, p(\rho) = 100\rho$}
$$\Vert g - \ln(\rho)\Vert$$
\begin{tabular}{*{5}{|c}|}
\hline
\diagbox{$\tau$}{$h$}&0.050000&0.025000&0.012500&0.0062500\\
\hline
0.050000&$5.564e+05$&$3.000e+00$&$3.000e+00$&$3.000e+00$\\
&$3.879e+05$&$3.267e+00$&$3.266e+00$&$3.266e+00$\\
&$4.257e+05$&$3.267e+00$&$3.266e+00$&$3.266e+00$\\
&$0.127$&$0.931$&$18.005$&$321.888$\\
\hline
0.025000&$5.270e+04$&$8.137e+03$&$1.504e+03$&$4.072e+14$\\
&$1.304e+04$&$2.681e+02$&$5.068e+02$&$6.254e+13$\\
&$1.692e+04$&$5.248e+02$&$5.188e+02$&$8.176e+13$\\
&$0.202$&$1.742$&$23.580$&$389.192$\\
\hline
0.012500&$3.499e+03$&$5.263e+13$&$7.901e+45$&$7.370e+12$\\
&$1.594e+02$&$1.993e+13$&$2.424e+45$&$2.427e+11$\\
&$2.756e+02$&$2.766e+13$&$3.583e+45$&$4.657e+11$\\
&$0.337$&$0.969$&$7.713$&$85.231$\\
\hline
0.0062500&$1.758e-02$&$1.135e+31$&$2.433e+05$&$2.717e+03$\\
&$1.809e-02$&$2.512e+30$&$2.377e+03$&$2.572e+01$\\
&$1.880e-02$&$4.452e+30$&$4.116e+03$&$4.351e+01$\\
&$0.295$&$2.064$&$82.523$&$585.628$\\
\hline
\end{tabular}
$$\Vert v_1 - u_1 \Vert$$
\begin{tabular}{*{5}{|c}|}
\hline
\diagbox{$\tau$}{$h$}&0.050000&0.025000&0.012500&0.0062500\\
\hline
0.050000&$0.000e+00$&$3.679e-01$&$3.679e-01$&$3.679e-01$\\
&$-nan$&$3.679e-01$&$3.679e-01$&$3.679e-01$\\
&$-nan$&$3.768e-01$&$3.701e-01$&$3.684e-01$\\
&$0.127$&$0.931$&$18.005$&$321.888$\\
\hline
0.025000&$2.266e+03$&$3.297e+00$&$2.045e+01$&$0.000e+00$\\
&$6.380e+02$&$2.437e+00$&$7.433e+00$&$-nan$\\
&$7.948e+02$&$2.460e+00$&$7.600e+00$&$-nan$\\
&$0.202$&$1.742$&$23.580$&$389.192$\\
\hline
0.012500&$3.131e+00$&$3.679e-01$&$3.679e-01$&$3.679e-01$\\
&$1.628e+00$&$3.679e-01$&$3.679e-01$&$3.679e-01$\\
&$1.817e+00$&$3.768e-01$&$3.701e-01$&$3.684e-01$\\
&$0.337$&$0.969$&$7.713$&$85.231$\\
\hline
0.0062500&$6.672e-01$&$3.679e-01$&$3.238e+00$&$2.569e+00$\\
&$3.868e-01$&$3.679e-01$&$1.596e+00$&$2.034e+00$\\
&$4.458e-01$&$3.768e-01$&$1.621e+00$&$2.041e+00$\\
&$0.295$&$2.064$&$82.523$&$585.628$\\
\hline
\end{tabular}
$$\Vert v_2 - u_2\Vert$$
\begin{tabular}{*{5}{|c}|}
\hline
\diagbox{$\tau$}{$h$}&0.050000&0.025000&0.012500&0.0062500\\
\hline
0.050000&$0.000e+00$&$2.718e+00$&$2.718e+00$&$2.718e+00$\\
&$-nan$&$2.718e+00$&$2.718e+00$&$2.718e+00$\\
&$-nan$&$2.784e+00$&$2.735e+00$&$2.722e+00$\\
&$0.127$&$0.931$&$18.005$&$321.888$\\
\hline
0.025000&$5.423e+01$&$2.696e+00$&$2.489e+01$&$0.000e+00$\\
&$1.690e+01$&$1.855e+00$&$1.146e+01$&$-nan$\\
&$2.027e+01$&$1.890e+00$&$1.161e+01$&$-nan$\\
&$0.202$&$1.742$&$23.580$&$389.192$\\
\hline
0.012500&$2.309e+00$&$2.718e+00$&$2.718e+00$&$2.718e+00$\\
&$1.153e+00$&$2.718e+00$&$2.718e+00$&$2.718e+00$\\
&$1.285e+00$&$2.784e+00$&$2.735e+00$&$2.722e+00$\\
&$0.337$&$0.969$&$7.713$&$85.231$\\
\hline
0.0062500&$6.632e-01$&$2.718e+00$&$3.156e+00$&$3.041e+00$\\
&$3.454e-01$&$2.718e+00$&$1.354e+00$&$1.286e+00$\\
&$3.999e-01$&$2.784e+00$&$1.383e+00$&$1.296e+00$\\
&$0.295$&$2.064$&$82.523$&$585.628$\\
\hline
\end{tabular}


\subsubsection{$\mu = 0.010, p(\rho) = 100\rho$}
$$\Vert g - \ln(\rho)\Vert$$
\begin{tabular}{*{5}{|c}|}
\hline
\diagbox{$\tau$}{$h$}&0.050000&0.025000&0.012500&0.0062500\\
\hline
0.050000&$3.507e+11$&$3.379e+01$&$2.277e+21$&$9.626e+23$\\
&$2.281e+11$&$2.396e+01$&$1.067e+21$&$4.221e+23$\\
&$2.621e+11$&$2.530e+01$&$1.109e+21$&$4.281e+23$\\
&$0.089$&$1.323$&$13.808$&$270.663$\\
\hline
0.025000&$3.411e+00$&$1.511e+11$&$2.577e+14$&$1.731e+16$\\
&$3.354e+00$&$6.551e+10$&$1.121e+14$&$7.398e+15$\\
&$3.393e+00$&$7.471e+10$&$1.175e+14$&$7.500e+15$\\
&$0.261$&$1.222$&$17.390$&$410.534$\\
\hline
0.012500&$2.950e+00$&$3.000e+00$&$3.012e+00$&$8.827e+06$\\
&$3.184e+00$&$3.267e+00$&$3.250e+00$&$4.582e+05$\\
&$3.209e+00$&$3.267e+00$&$3.251e+00$&$6.195e+05$\\
&$0.266$&$2.143$&$26.126$&$442.334$\\
\hline
0.0062500&$1.694e-02$&$4.729e+25$&$5.811e+13$&$1.213e+68$\\
&$1.701e-02$&$9.173e+23$&$1.564e+13$&$3.809e+66$\\
&$1.828e-02$&$1.604e+24$&$1.684e+13$&$6.660e+66$\\
&$0.262$&$1.060$&$9.596$&$220.156$\\
\hline
\end{tabular}
$$\Vert v_1 - u_1 \Vert$$
\begin{tabular}{*{5}{|c}|}
\hline
\diagbox{$\tau$}{$h$}&0.050000&0.025000&0.012500&0.0062500\\
\hline
0.050000&$0.000e+00$&$2.855e+17$&$0.000e+00$&$0.000e+00$\\
&$-nan$&$3.312e+16$&$-nan$&$-nan$\\
&$-nan$&$3.870e+16$&$-nan$&$-nan$\\
&$0.089$&$1.323$&$13.808$&$270.663$\\
\hline
0.025000&$4.000e+01$&$0.000e+00$&$0.000e+00$&$0.000e+00$\\
&$1.045e+01$&$-nan$&$-nan$&$-nan$\\
&$1.352e+01$&$-nan$&$-nan$&$-nan$\\
&$0.261$&$1.222$&$17.390$&$410.534$\\
\hline
0.012500&$1.359e+01$&$3.679e-01$&$2.902e+00$&$8.236e+09$\\
&$7.872e+00$&$3.679e-01$&$2.909e+00$&$2.192e+08$\\
&$8.545e+00$&$3.768e-01$&$2.943e+00$&$2.556e+08$\\
&$0.266$&$2.143$&$26.126$&$442.334$\\
\hline
0.0062500&$1.172e+00$&$3.679e-01$&$3.679e-01$&$3.679e-01$\\
&$5.730e-01$&$3.679e-01$&$3.679e-01$&$3.679e-01$\\
&$7.164e-01$&$3.768e-01$&$3.701e-01$&$3.684e-01$\\
&$0.262$&$1.060$&$9.596$&$220.156$\\
\hline
\end{tabular}
$$\Vert v_2 - u_2\Vert$$
\begin{tabular}{*{5}{|c}|}
\hline
\diagbox{$\tau$}{$h$}&0.050000&0.025000&0.012500&0.0062500\\
\hline
0.050000&$0.000e+00$&$1.054e+08$&$0.000e+00$&$0.000e+00$\\
&$-nan$&$1.949e+07$&$-nan$&$-nan$\\
&$-nan$&$2.240e+07$&$-nan$&$-nan$\\
&$0.089$&$1.323$&$13.808$&$270.663$\\
\hline
0.025000&$3.311e+01$&$0.000e+00$&$0.000e+00$&$0.000e+00$\\
&$1.046e+01$&$-nan$&$-nan$&$-nan$\\
&$1.384e+01$&$-nan$&$-nan$&$-nan$\\
&$0.261$&$1.222$&$17.390$&$410.534$\\
\hline
0.012500&$1.256e+01$&$2.718e+00$&$2.513e+00$&$6.106e+43$\\
&$7.582e+00$&$2.718e+00$&$2.681e+00$&$1.132e+42$\\
&$8.864e+00$&$2.784e+00$&$2.701e+00$&$1.478e+42$\\
&$0.266$&$2.143$&$26.126$&$442.334$\\
\hline
0.0062500&$1.328e+00$&$2.718e+00$&$2.718e+00$&$2.718e+00$\\
&$5.700e-01$&$2.718e+00$&$2.718e+00$&$2.718e+00$\\
&$7.140e-01$&$2.784e+00$&$2.735e+00$&$2.722e+00$\\
&$0.262$&$1.060$&$9.596$&$220.156$\\
\hline
\end{tabular}


\subsubsection{$\mu = 0.001, p(\rho) = 100\rho$}
$$\Vert g - \ln(\rho)\Vert$$
\begin{tabular}{*{5}{|c}|}
\hline
\diagbox{$\tau$}{$h$}&0.050000&0.025000&0.012500&0.0062500\\
\hline
0.050000&$1.903e+14$&$5.326e+01$&$7.837e+40$&$2.795e+56$\\
&$1.195e+14$&$2.832e+01$&$3.230e+40$&$1.011e+56$\\
&$1.386e+14$&$3.063e+01$&$3.428e+40$&$1.039e+56$\\
&$0.080$&$1.634$&$26.152$&$284.902$\\
\hline
0.025000&$8.198e+15$&$3.020e+04$&$1.522e+31$&$5.647e+65$\\
&$2.246e+15$&$3.764e+03$&$6.005e+30$&$1.439e+65$\\
&$2.987e+15$&$4.431e+03$&$6.438e+30$&$1.593e+65$\\
&$0.226$&$2.377$&$61.384$&$253.979$\\
\hline
0.012500&$6.202e+13$&$1.081e+01$&$3.494e+00$&$2.937e+24$\\
&$3.196e+13$&$4.199e+00$&$3.251e+00$&$2.324e+23$\\
&$5.960e+13$&$4.793e+00$&$3.265e+00$&$3.607e+23$\\
&$0.203$&$1.473$&$37.439$&$360.658$\\
\hline
0.0062500&$1.819e-02$&$1.407e+10$&$2.988e+00$&$2.723e+01$\\
&$1.719e-02$&$1.333e+09$&$3.241e+00$&$5.833e+00$\\
&$1.870e-02$&$1.901e+09$&$3.242e+00$&$6.537e+00$\\
&$0.259$&$1.577$&$42.052$&$310.570$\\
\hline
\end{tabular}
$$\Vert v_1 - u_1 \Vert$$
\begin{tabular}{*{5}{|c}|}
\hline
\diagbox{$\tau$}{$h$}&0.050000&0.025000&0.012500&0.0062500\\
\hline
0.050000&$0.000e+00$&$9.703e+25$&$0.000e+00$&$0.000e+00$\\
&$-nan$&$1.034e+25$&$-nan$&$-nan$\\
&$-nan$&$1.227e+25$&$-nan$&$-nan$\\
&$0.080$&$1.634$&$26.152$&$284.902$\\
\hline
0.025000&$3.679e-01$&$2.822e+20$&$0.000e+00$&$3.679e-01$\\
&$3.679e-01$&$8.452e+18$&$-nan$&$3.679e-01$\\
&$4.023e-01$&$1.483e+19$&$-nan$&$3.684e-01$\\
&$0.226$&$2.377$&$61.384$&$253.979$\\
\hline
0.012500&$3.679e-01$&$4.745e+04$&$1.113e+02$&$1.380e+18$\\
&$3.679e-01$&$4.668e+03$&$1.225e+01$&$1.356e+17$\\
&$4.023e-01$&$8.133e+03$&$1.809e+01$&$1.447e+17$\\
&$0.203$&$1.473$&$37.439$&$360.658$\\
\hline
0.0062500&$1.256e+00$&$3.679e-01$&$7.081e+00$&$6.439e+05$\\
&$6.115e-01$&$3.679e-01$&$2.875e+00$&$3.260e+04$\\
&$7.848e-01$&$3.768e-01$&$3.182e+00$&$5.476e+04$\\
&$0.259$&$1.577$&$42.052$&$310.570$\\
\hline
\end{tabular}
$$\Vert v_2 - u_2\Vert$$
\begin{tabular}{*{5}{|c}|}
\hline
\diagbox{$\tau$}{$h$}&0.050000&0.025000&0.012500&0.0062500\\
\hline
0.050000&$0.000e+00$&$1.686e+12$&$0.000e+00$&$0.000e+00$\\
&$-nan$&$2.430e+11$&$-nan$&$-nan$\\
&$-nan$&$2.898e+11$&$-nan$&$-nan$\\
&$0.080$&$1.634$&$26.152$&$284.902$\\
\hline
0.025000&$2.718e+00$&$2.661e+20$&$0.000e+00$&$2.718e+00$\\
&$2.718e+00$&$9.953e+18$&$-nan$&$2.718e+00$\\
&$2.972e+00$&$1.636e+19$&$-nan$&$2.722e+00$\\
&$0.226$&$2.377$&$61.384$&$253.979$\\
\hline
0.012500&$2.718e+00$&$1.803e+03$&$4.463e+01$&$0.000e+00$\\
&$2.718e+00$&$1.224e+02$&$1.069e+01$&$-nan$\\
&$2.972e+00$&$2.207e+02$&$1.459e+01$&$-nan$\\
&$0.203$&$1.473$&$37.439$&$360.658$\\
\hline
0.0062500&$1.410e+00$&$2.718e+00$&$5.855e+00$&$1.833e+06$\\
&$6.135e-01$&$2.718e+00$&$3.178e+00$&$2.690e+04$\\
&$7.870e-01$&$2.784e+00$&$3.542e+00$&$4.469e+04$\\
&$0.259$&$1.577$&$42.052$&$310.570$\\
\hline
\end{tabular}




%\include{task2}
%
%\include{task3}
%
%\include{task4}

\end{document}
